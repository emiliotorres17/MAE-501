\section{Exercise 2}
Let $A$ be matrix
\begin{equation}
    A =
    \begin{bmatrix}
        1   &   2   &   3   &   1   \\
        1   &   4   &   3   &   2   \\
        -1   &   0   &   1   &   2
    \end{bmatrix}
\end{equation}
\begin{enumerate}[label=(\alph*)]
    \item (4pts) Calculate the $LU$ factorization of $A$.
        \begin{mdframed}[style=MyFrame]
            The $LU$ factorization can be obtained by first augmenting $A$
            with a $3 \times 3$ identity matrix and performing row
            operations to produce an upper triangular matrix,
            \begin{equation}
                \begin{bmatrix}[c|c]
                    A   &   I
                \end{bmatrix}
                =
                \begin{bmatrix}[cccc|ccc]
                    1   &   2   &   3   &   1   &   1   &   0   &   0   \\
                    1   &   4   &   3   &   2   &   0   &   1   &   0   \\
                    -1  &   0   &   1   &   2   &   0   &   0   &   1   
                \end{bmatrix}
            \end{equation}
            Performing $r_{2} = r_{2}-r_{1}$ and $r_{3} = r_{3} + r_{1}$
            gives,
            \begin{equation}
                \begin{bmatrix}[cccc|ccc]
                    1   &   2   &   3   &   1   &   1   &   0   &   0   \\
                    0   &   2   &   0   &   1   &   -1  &   1   &   0   \\
                    0   &   2   &   4   &   3   &   1   &   0   &   1   
                \end{bmatrix}
            \end{equation}
            Next $r_{3} = r_{3}-r_{2}$
            \begin{equation}
                \begin{bmatrix}[cccc|ccc]
                    1   &   2   &   3   &   1   &   1   &   0   &   0   \\
                    0   &   2   &   0   &   1   &   -1  &   1   &   0   \\
                    0   &   0   &   4   &   2   &   1   &   -1  &   1   
                \end{bmatrix}
            \end{equation}
            Furthermore, $L$ can be obtained using the operations, namely
            \begin{equation}
                L^{-1} = E_{32}E_{31}E_{21}
            \end{equation}
            Therefore,
            \begin{equation}
                L = E_{21}^{-1} E_{31}^{-1} E_{32}^{-1}
            \end{equation}
            Thus,
            \begin{equation}
                U = 
                \begin{bmatrix}
                    1   &   2   &   3   &   1   \\
                    0   &   2   &   0   &   1   \\
                    0   &   0   &   4   &   2
                \end{bmatrix}
            \end{equation}
            and 
            \begin{equation}
                L =
                \begin{bmatrix}
                    1   &   0   &   0   \\
                    1   &   1   &   0   \\
                    -1  &   1   &   1
                \end{bmatrix}
            \end{equation}
        \end{mdframed}
    \item (1pt) What is the rank of $A$?
        \begin{mdframed}[style=MyFrame]
            Since $A$ has 3 pivots $\text{rank}(A)=3$.
        \end{mdframed}
    \item (2pts) What are the dimensions of the four fundamental subspaces
        of $A$?
        \begin{mdframed}[style=MyFrame]
            \begin{equation}
                \text{dim}(C(A^{T})) = 3
            \end{equation}
            \begin{equation}
                \text{dim}(N(A)) = 4-3 = 1
            \end{equation}
            \begin{equation}
                \text{dim}(C(A)) = 3
            \end{equation}
            \begin{equation}
                \text{dim}(N(A^{T})) = 3-3 = 0
            \end{equation}
        \end{mdframed}

    \item (3pts) Give the complete solution to $Ax=b$ with
        \begin{equation}
            b = 
            A 
            \begin{bmatrix}
                1   \\
                2   \\
                3   \\
                4
            \end{bmatrix}
        \end{equation}
        \begin{mdframed}[style=MyFrame]
            In order to find the complete solutions to the system we need
            to find both the particular and special solutions. Clearly,
            from the problem statement we can express the particular
            solution as
            \begin{equation}
                x_{p} =
                \begin{bmatrix}
                    1       \\
                    2       \\
                    3       \\
                    4
                \end{bmatrix}
            \end{equation}
            Next we can find the \emph{\textbf{one}} special solution using
            the nullspace of $A$ which we can find from the row reduced
            echelon form, 
            \begin{equation}
                \text{rref}(A) =
                \begin{bmatrix}
                    1       &   0   &   0   &   -3/2    \\
                    0       &   1   &   0   &   1/2     \\
                    0       &   0   &   1   &   1/2
                \end{bmatrix}
            \end{equation}
            Therefore,
            \begin{equation}
                N(A) =
                \begin{bmatrix}
                    -3      \\
                    1       \\
                    1       \\
                    -2
                \end{bmatrix}
            \end{equation}
            Thus, the complete solution is
            \begin{equation}
                \begin{bmatrix}
                    1       \\
                    2       \\
                    3       \\
                    4
                \end{bmatrix}
                +
                \alpha
                \begin{bmatrix}
                    -3      \\
                    1       \\
                    1       \\
                    -2
                \end{bmatrix}
            \end{equation}
        \end{mdframed}
\end{enumerate}






                    
