\section{Exercise 1}
Given the matrix 
\begin{equation}
    A = 
    \begin{bmatrix}
        0       &       0       &       6       \\
        1       &       2       &       3       \\
        0       &       4       &       5
    \end{bmatrix}
\end{equation}

\begin{enumerate}[label=(\alph*)]
    \item (4pts) Find the permutation $P$ that makes $PA$ upper triangular.
        (Explain you process)

        \begin{mdframed}[style=MyFrame]
            We start by swapping $r_{1}$ and $r_{2}$ using,
            \begin{equation}
                P_{21} =
                \begin{bmatrix}
                    0       &       1       &       0       \\
                    1       &       0       &       1       \\
                    0       &       0       &       1
                \end{bmatrix}
            \end{equation}
            which gives
            \begin{equation}
                P_{21}A = 
                \begin{bmatrix}
                    1       &       2       &       3       \\
                    0       &       0       &       6       \\
                    0       &       4       &       5
                \end{bmatrix}
            \end{equation}
            Next we can use
            \begin{equation}
                P_{23} =
                \begin{bmatrix}
                    1       &       0       &       0       \\
                    0       &       0       &       1       \\
                    0       &       1       &       0
                \end{bmatrix}
            \end{equation}
            to switch $r_{2}$ and $r_{3}$ giving, 
            \begin{equation}
                P_{23} P_{21} A = 
                \begin{bmatrix}
                    1       &       2       &       3       \\
                    0       &       4       &       5       \\
                    0       &       0       &       6  
                \end{bmatrix}
            \end{equation}
            Thus,
            \begin{equation}
                P = P_{23}P_{21} = 
                \begin{bmatrix}
                    0       &       1       &       0       \\
                    0       &       0       &       1       \\
                    1       &       0       &       0       
                \end{bmatrix}
            \end{equation}
        \end{mdframed}
    \item (2pts) Is $PA$ invertible? Explain why.
        \begin{mdframed}[style=MyFrame]
            Yes, since $PA$ is an upper triangular matrix with non-zero
            diagonal entries means that it is of full rank.
        \end{mdframed}
    \item (2pts) Using properties of permutation matrices, find the inverse
        of $P$.
        \begin{mdframed}[style=MyFrame]
            Using $P^{-1}=P^{T}$ we can easily obtain the inverse of $P$,
            namely 
            \begin{equation}
                P^{-1} = P^{T} = 
                \begin{bmatrix}
                    0       &       0       &       1       \\
                    1       &       0       &       0       \\
                    0       &       1       &       0       \\
                \end{bmatrix}
            \end{equation}
        \end{mdframed}
    \item (2pts) Calculate $A^{-1}$.
        \begin{mdframed}[style=MyFrame]
            We can easily find the inverse of $A$ using 
            \begin{equation}
                \left(PA\right)^{-1} = A^{-1} P^{-1}
            \end{equation}
            Solving $A^{-1}$ we get 
            \begin{equation}
                A^{-1} = \left(PA\right)^{-1}P
            \end{equation}
            Next we can use an augmented matrix approach to determine
            $(PA)^{-1}$, namely
            \begin{equation}
                \begin{bmatrix}[c|c]
                    PA  &   I
                \end{bmatrix}
                = 
                \begin{bmatrix}[ccc|ccc]
                    1   &   2   &   3   &   1   &   0   &   0   \\
                    0   &   4   &   5   &   0   &   1   &   0   \\
                    0   &   0   &   6   &   0   &   0   &   1
                \end{bmatrix}
            \end{equation}
            Furthermore, after performing all the backwards eliminations 
            and dividing through by the trace values we get the following
            \begin{equation}
                (PA)^{-1}   =
                    \begin{bmatrix}
                        1       &       -1/2        &       -1/12       \\
                        0       &       1/4         &       -5/24       \\
                        0       &       0           &       1/6
                    \end{bmatrix}
            \end{equation}
            Thus,
            \begin{equation}
                A^{-1} = (PA)^{-1}P =
                \begin{bmatrix}
                    -1/12       &       1       &       -1/2        \\
                    -5/24       &       0       &       1/4         \\
                    1/6         &       0       &       0           
                \end{bmatrix}
            \end{equation}
        \end{mdframed}
\end{enumerate}
