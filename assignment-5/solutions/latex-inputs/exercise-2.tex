\section{Exercise 2}
\begin{enumerate}[label=\arabic*.]
    \item Factor $A$ into $S \Lambda S^{-1}$:
        \begin{equation}
            A = 
            \begin{bmatrix}
                1 & 2 \\
                0 & 3
            \end{bmatrix}
        \end{equation}
    If $A=S \Lambda S^{-1}$ then $A^{3} = (\;\;\;)(\;\;\;)(\;\;\;)$ and
        $A^{-1} = (\;\;\;)(\;\;\;)(\;\;\;)$?
        \begin{mdframed}[style=MyFrame]
            We start by finding the eigenvalues of $A$ which are
            $\lambda_{1} =  3$ and $\lambda_{2}=1$. Next, we can find the
            eigenvectors by solving
            \begin{equation}
                A - \lambda I = 0 
            \end{equation}
            which gives the eigenvectors,
            \begin{equation}
                v_{1} = 
                \begin{bmatrix}
                    1   \\
                    1
                \end{bmatrix}
            \end{equation}
            and
            \begin{equation}
                v_{2}  =
                \begin{bmatrix}
                    1   \\
                    0
                \end{bmatrix}
            \end{equation}
            This gives the following factors of $A$,
            \begin{equation}
                A = 
                \underbrace{
                \begin{bmatrix}
                    1       &   1   \\
                    0       &   1   \\
                \end{bmatrix}
                }_{S}
                \underbrace{
                \begin{bmatrix}
                    1       &   0   \\
                    0       &   3   \\
                \end{bmatrix}
                }_{\Lambda}
                \underbrace{
                \begin{bmatrix}
                    1       &   -1  \\
                    0       &   1   \\
                \end{bmatrix}
            }_{S^{-1}}
            \end{equation}
            Therefore we can find $A^{3}$ using the following 
            \begin{equation}
                A^{3} = 
                \begin{bmatrix}
                    1       &   1   \\
                    0       &   1   \\
                \end{bmatrix}
                \begin{bmatrix}
                    1^{3}   &   0       \\
                    0       &   3^{3}   \\
                \end{bmatrix}
                \begin{bmatrix}
                    1       &   -1  \\
                    0       &   1   \\
                \end{bmatrix}
                =
                \begin{bmatrix}
                    1           &       26  \\
                    0           &       27
                \end{bmatrix}
            \end{equation}
            Lastly, we cand find $A^{-1}$ using the same methodology as
            above,
            \begin{equation}
                A^{-1} = 
                \begin{bmatrix}
                    1       &   1   \\
                    0       &   1   \\
                \end{bmatrix}
                \begin{bmatrix}
                    1^{-1}   &   0       \\
                    0        &   3^{-1}  \\
                \end{bmatrix}
                \begin{bmatrix}
                    1       &   -1  \\
                    0       &   1   \\
                \end{bmatrix}
                =
                \begin{bmatrix}
                    1           &      -2/3  \\
                    0           &       1/3
                \end{bmatrix}
            \end{equation}
        \end{mdframed}

    \item Diagonalize $A$ and compute $S \Lambda^{k} S^{-1}$ to prove the
        following formula for $A^{k}$:
        \begin{equation}
            A =
            \begin{bmatrix}
                2 & -1\\
                -1 & 2
            \end{bmatrix}
        \end{equation}
        and 
        \begin{equation}
            A^{k} =
            \frac{1}{2}
            \begin{bmatrix}
                1 + 3^{k} & 1-3^{k} \\
                1-3^{k} & 1+3^{k}
            \end{bmatrix}
        \end{equation}
        \begin{mdframed}[style=MyFrame]
            Similarly to above we start by finding the eigenvalues of $A$
            which are $\lambda_{1}=3$ and $\lambda_{2} = 1$. Next we can
            use the eigenvalues to find the eigenvectors by solving
            \begin{equation}
                \left( A - \lambda I\right)\mathbf{v} = 0
            \end{equation}
            giving
            \begin{equation}
                v_{1} = 
                \begin{bmatrix}
                    1       \\
                    -1      
                \end{bmatrix}
            \end{equation}
            and 
            \begin{equation}
                v_{2} = 
                \begin{bmatrix}
                    1       \\
                    1      
                \end{bmatrix}
            \end{equation}
            Therefore we can factor $A$ into the following
            \begin{equation}
                A = 
                \underbrace{
                    \begin{bmatrix}
                        1       &   1       \\
                        -1      &   1
                    \end{bmatrix}
                }_{S}
                \underbrace{
                    \begin{bmatrix}
                        3       &   0       \\
                        0       &   1
                    \end{bmatrix}
                }_{\Lambda}
                \underbrace{
                    \begin{bmatrix}
                        1/2     &   -1/2      \\
                        1/2     &   1/2
                    \end{bmatrix}
                }_{S^{-1}}
            \end{equation}
            Lastly we can verify the formula by computing $A^{5}$ using the
            factors,
            \begin{equation}
                A^{5} = 
                    \begin{bmatrix}
                        1       &   1       \\
                        -1      &   1
                    \end{bmatrix}
                    \begin{bmatrix}
                        3^{5}       &   0       \\
                        0           &   1^{5}
                    \end{bmatrix}
                    \begin{bmatrix}
                        1/2     &   -1/2      \\
                        1/2     &   1/2
                    \end{bmatrix}
                    =
                    \begin{bmatrix}
                        122         &       -121        \\
                        -121        &       122
                    \end{bmatrix}
            \end{equation}
            From the formula we get 
            \begin{equation}
                A^{5} = 
                \frac{1}{2}
                \begin{bmatrix}
                    244     &       -242        \\
                    -242    &       244
                \end{bmatrix}
                =
                \begin{bmatrix}
                    122         &       -121        \\
                    -121        &       122
                \end{bmatrix}
            \end{equation}
            Clearly the two solutions match and we can verify that formula
            for $A^{k}$ is correct.
        \end{mdframed}

    \item Find the eigenvalues and eigenvectors of $A$ and write
        $\mathbf{u}(0)=(0, 2\sqrt{2}, 0)$ as a combination of the
        eigenvectors. Solve both equations $\mathbf{u}^{\prime} = A
        \mathbf{u}$ and $\mathbf{u}^{\prime \prime} = A u$:
    \begin{equation}
        \frac{d\mathbf{u}}{dt} = 
        \begin{bmatrix}
            -2 & 1 & 0 \\
            1 & -2 & 1 \\
            0 &1 & -2
        \end{bmatrix}
        \mathbf{u}
    \end{equation}
    and 
    \begin{equation}
        \frac{d^{2}\mathbf{u}}{dt^{2}} = 
        \begin{bmatrix}
            -2 & 1 & 0 \\
            1 & -2 & 1 \\
            0 &1 & -2
        \end{bmatrix}
        \mathbf{u}
    \end{equation}
    with
    \begin{equation}
        \frac{d\mathbf{u}}{dt}(0) = 0
    \end{equation}
    \begin{mdframed}[style=MyFrame]
        We start by finding the eigenvalues and eigenvectors from
        $\deter(A-\lambda I)$, namely
        \begin{equation}
            \deter\left(A-\lambda I\right) =
            \deter
            \begin{bmatrix}
                -2-\lambda      &   1           &   0           \\
                1               &   -2-\lambda  &   1           \\
                0               &   1           &   -2-\lambda 
            \end{bmatrix}
            =
            0
        \end{equation}
        This gives the characteristic polynomial,
        \begin{equation}
            \left(-2-\lambda\right)\left[\left(-2-\lambda\right)-2\right] =
            0
        \end{equation}
        Thus $\lambda_{1}=-2$, $\lambda_{2}=-2-\sqrt{2}$, and
        $\lambda_{3}=-2+\sqrt{2}$. Next, we can find the eigenvectors using
        \begin{equation}
            \left( A-\lambda I\right) \mathbf{x} = 0
        \end{equation}
        which gives,
        \begin{equation}
            x_{1} = 
            \begin{bmatrix}
                1       \\
                0       \\
                -1
            \end{bmatrix}
        \end{equation}
        \begin{equation}
            x_{2} = 
            \begin{bmatrix}
                1           \\
                -\sqrt{2}   \\
                1
            \end{bmatrix}
        \end{equation}
        and
        \begin{equation}
            x_{3} = 
            \begin{bmatrix}
                1           \\
                \sqrt{2}   \\
                1
            \end{bmatrix}
        \end{equation}
        Since the eigenvectors for symmetric matrix are always orthogonal
        and because all three eigenvalues are negative means that $A$ is
        negative definite and $e^{At}$ decays to zero. Lastly we can see
        that the initial conditions is satisfied with $x_{3}-x_{2}$ and
        therefore the solution is $u(t)=e^{\lambda_{3}t}x_{3} -
        e^{\lambda_{2}t}x_{2}$.

    \end{mdframed}

    \item The rabbit population shows fast growth ($6r$) but loss to wolves
        ($-2w$). The wolf population always grows in this model:
    \begin{equation}
        \frac{dr}{dt}=6r-2w
    \end{equation}
    and
    \begin{equation}
        \frac{dw}{dt}=2r+w
    \end{equation}
    Find the eigenvalues and eigenvectors. If $r(0)=w(0)=30$ what are the
    populations at time $t$? What is the ratio of rabbits to wolves?
    \begin{mdframed}[style=MyFrame]
        WE start by expressing the system of ordinary differential
        equations in matrix form 
        \begin{equation}
            u^{\prime} =
            \underbrace{
            \begin{bmatrix}
                6       &       -2  \\
                2       &       1
            \end{bmatrix}
        }_{\equiv A}
            u
        \end{equation}
        where
        \begin{equation}
            u = 
            \begin{bmatrix}
                r   \\
                w
            \end{bmatrix}
        \end{equation}
        Next, we can find the eigenvalues of of matrix $A$,
        \begin{equation}
            \deter\left(A-\lambda I\right) =
            \deter
            \begin{bmatrix}
                6-\lambda       &       -2              \\
                2               &       1-\lambda
            \end{bmatrix}
        \end{equation}
        This gives the following characteristic polynomial
        \begin{equation}
            \lambda^{2} - 7\lambda + 10 = 0
        \end{equation}
        Thus $\lambda_{1} = 5$ and $\lambda_{2} = 2$. Next we can use
        $(A-\lambda)x$ to obtain the eigenvectors,
        \begin{equation}
            x_{1} = 
            \begin{bmatrix}
                2   \\
                1
            \end{bmatrix}
        \end{equation}
        and 
        \begin{equation}
            x_{2} = 
            \begin{bmatrix}
                1   \\
                2
            \end{bmatrix}
        \end{equation}
        This gives the general solution as
        \begin{equation}
            u(t) = c_{1}e^{5t}
            \begin{bmatrix}
                2   \\
                1
            \end{bmatrix}
            + c_{2}e^{2t}
            \begin{bmatrix}
                1   \\
                2
            \end{bmatrix}
        \end{equation}
        Moreover we can apply the initial conditions to find the two
        constants,
        \begin{equation}
            u(0) = 
            \begin{bmatrix}
                30  \\
                30
            \end{bmatrix}
            =
            \begin{bmatrix}
                2c_{1} + c_{2}  \\
                c_{1} + 2c_{2}
            \end{bmatrix}
        \end{equation}
        Solving for the constants we get $c_{1}=10$ and $c_{2}=10$,
        therefore we get, 
        \begin{equation}
            r(t) = 20e^{5t} + 10e^{2t} 
        \end{equation}
        and
        \begin{equation}
            w(t) = 10e^{5t} + 20e^{2t}
        \end{equation}
        Lastly if we take the limit as time approach infintiy,
        \begin{equation}
            \lim_{t \to \infty} \frac{r(t)}{w(t)}
            =
            \lim_{t \to \infty} 
                \frac{20e^{5t} + 10e^{2t}}
                {10e^{5t} + 20e^{2t}} 
                = \frac{20}{10}
        \end{equation}
        we can see that the rabbits will out number the wolves 2 to 1.
    \end{mdframed}
\end{enumerate}
