\section{Exercise 3}
\begin{enumerate}[label=\arabic*.]
    \item Find the eigenvalues and unit eigenvectors $\mathbf{v}_{1}$,
        $\mathbf{v}_{2}$ of $A^{T}A$. Then find $\mathbf{u}_{1} =
        A\mathbf{v}_{1}/\mathbf{\sigma}_{1}$:
        \begin{equation}
            A=
            \begin{bmatrix}
            1 & 2\\
            3 & 6
            \end{bmatrix}
        \end{equation}
        Verify that $\mathbf{u}_{1}$ is a unit eigenvector of $AA^{T}$.
        Complete the matrices $U$, $\Sigma$, $V$.
        \begin{mdframed}[style=MyFrame]
            We start by first finding $A^{T}A$,
            \begin{equation}
                A^{T}A  = 
                \begin{bmatrix}
                    10      &       20      \\
                    20      &       40 
                \end{bmatrix}
            \end{equation}
            which has eigenvalues $\lambda_{1} = 50$ and $\lambda_{2} = 0$.
            The using $\deter\left(A^{T}A-\lambda I\right)x=0$ to obtain
            the following eigenvectors,
            \begin{equation}
                v_{1} = 
                \frac{1}{\sqrt{5}}
                \begin{bmatrix}
                    1       \\
                    2
                \end{bmatrix}
            \end{equation}
            and
            \begin{equation}
                v_{2} = 
                \frac{1}{\sqrt{5}}
                \begin{bmatrix}
                    -2      \\
                    1
                \end{bmatrix}
            \end{equation}
            Moreover we can use $\sigma_{i}^{2}=\lambda_{i}$ to calculate
            the singular values, namely 
            \begin{equation}
                \sigma_{1} =  \sqrt{50}
            \end{equation}
            and
            \begin{equation}
                \sigma_{2}  = 0 
            \end{equation}
            Next we can use the eigenvectors of $AA^{T}$ to determine
            $\mathbf{U}$, giving
            \begin{equation}
                u_{1} = 
                \frac{1}{\sqrt{10}}
                \begin{bmatrix}
                    1       \\
                    3
                \end{bmatrix}
            \end{equation}
            and
            \begin{equation}
                u_{2} =
                \frac{1}{\sqrt{10}}
                \begin{bmatrix}
                    3       \\
                    -1
                \end{bmatrix}
            \end{equation}
            This gives the factors of $A$ as,
            \begin{equation}
                A = 
                \underbrace{
                    \frac{1}{\sqrt{10}}
                    \begin{bmatrix}
                        1       &       3       \\
                        3       &       -1
                    \end{bmatrix}
                }_{\equiv U}
                \underbrace{
                    \begin{bmatrix}
                        \sqrt{50}       &       0       \\
                        0               &       0
                    \end{bmatrix}
                }_{\equiv \Sigma}
                \underbrace{
                    \frac{1}{\sqrt{5}}
                    \begin{bmatrix}
                        1               &       2       \\
                        2               &       -1
                    \end{bmatrix}
                }_{\equiv V^{T}}
            \end{equation}
        \end{mdframed}
        
    \item Compute $A^{T}A$ and $AA^{T}$ and their eigenvalues and unit
        eigenvectors for the rectangular matrix A:
        \begin{equation}
            A = 
            \begin{bmatrix}
                1 & 1 & 0 \\
                0 & 1 & 1
            \end{bmatrix}
        \end{equation}
        Multiply the matrices $U \Sigma V^{T}$ to recover $A$. 
        \begin{mdframed}[style=MyFrame]
            Starting with $A^{T}A$
            \begin{equation}
                A^{T}A = 
                \begin{bmatrix}
                    1       &       1   &       0       \\
                    1       &       2   &       1       \\
                    0       &       1   &       1
                \end{bmatrix}
            \end{equation}
            and finding the eigenvalues with
            $\deter\left(A^{T}T-\lambda I\right)$ which gives $\lambda_{1} =
            3$, $\lambda_{2} = 1$, and $\lambda_{3} = 0$. Next, we can use
            $\deter\left(A^{T}A - \lambda I\right)x = 0$ (also need to
            normalize) to find the unit eigenvectors giving 
            \begin{equation}
                x_{1} = 
                \frac{1}{\sqrt{6}}
                \begin{bmatrix}
                    1       \\
                    2       \\
                    1
                \end{bmatrix}
            \end{equation}
            \begin{equation}
                x_{2} =
               \frac{1}{\sqrt{2}}
                \begin{bmatrix}
                    -1      \\
                    0       \\
                    1
                \end{bmatrix}
            \end{equation}
            and
            \begin{equation}
                x_{3} =
                \frac{1}{\sqrt{3}}
                \begin{bmatrix}
                    1       \\
                    -1      \\
                    1
                \end{bmatrix}
            \end{equation}
            Moreover we can use $\norm{Av_{i}}$ to find the two singular
            values, namely
            \begin{equation}
                \sigma_{1} = \norm{Av_{1}} = \sqrt{3}
            \end{equation}
            \begin{equation}
                \sigma_{2} = \norm{Av_{2}} = 1
            \end{equation}
            Additionally we can perform the same for $AA^{T}$,
            \begin{equation}
                AA^{T} =
                \begin{bmatrix}
                    2       &       1       \\
                    1       &       2
                \end{bmatrix}
            \end{equation}
            which has the eigenvalues $\lambda_{1} = 3$ and $\lambda_{2} =
            1$. Again we can use $\deter\left(AA^{T} - \lambda I\right)x =
            0$ (also need to normalize) to find unit eigenvectors giving
            \begin{equation}
                u_{1} = 
                \frac{1}{\sqrt{2}}
                \begin{bmatrix}
                    1       \\
                    1
                \end{bmatrix}
            \end{equation}
            \begin{equation}
                u_{2} =
                \frac{1}{\sqrt{2}}
                \begin{bmatrix}
                    -1      \\
                    1
                \end{bmatrix}
            \end{equation}
            Thus we can factor $A$ into 
            \begin{equation}
                A = 
                \underbrace{
                    \begin{bmatrix}
                        1       &       -1      \\
                        1       &       1       
                    \end{bmatrix}
                }_{\equiv U}
                \underbrace{
                    \begin{bmatrix}
                        \sqrt{3}        &       0       &       0       \\
                        0               &       1       &       0       
                    \end{bmatrix}
                }_{\equiv \Sigma}
                \underbrace{
                    \begin{bmatrix}
                        1/\sqrt{6}      &       2/\sqrt{6}      &   1/\sqrt{6}          \\
                        -1/\sqrt{2}     &       0               &   1/\sqrt{2}          \\
                        1/\sqrt{3}      &       -1/\sqrt{3}     &   1/\sqrt{3}
                    \end{bmatrix}
                }_{\equiv V^{T}}
            \end{equation}
        \end{mdframed}
    \item Suppose $A$ is a 2 by 2 symmetric matrix with unit eigenvectors
        $\mathbf{u}_{1}$ and $\mathbf{u}_{2}$. If its eigenvalues are
        $\lambda_{1} = 3$ and $\lambda_{2} = -2$ what are the matrices $U$,
        $\Sigma$, and $V^{T}$ in its SVD?
        \begin{mdframed}[style=MyFrame]
            Since $A$ is a symmetric matrix then $A^{T}=A$ then
            $\sigma_{1}^{2} = \lambda_{1}^{2}$ and $\sigma_{2}^{2} =
            \lambda_{2}^{2}$. However because of the squaring term the
            negative sign is lost therefore we get $\sigma_{1}=3$ and
            $\sigma_{2}=2$. Furthermore the unit eigenvectors for $AA^{T}$
            which are $u_{1}$ and $u_{2}$ are the same for $A^{T}A$ except
            for the sign change due to squaring term, thus we get $v_{1} =
            u_{1}$ and $v_{2}=-u_{2}$.   
        \end{mdframed}
    \item If $A=QR$ with an orthogonal matrix $Q$, the SVD of $A$ is almost
        the same as the SVD of $R$. Which of the three matrices $U$,
        $\Sigma$, $V$ is changed because of $Q$?
        \begin{mdframed}[style=MyFrame]
            If we assume that the SVD of $A$ is $R$ then $R=U \Sigma V^{T}$.
            Therefore to recover $A$ we still have to multiply $R$ by $Q$,
            giving
            \begin{equation}
                A = QR = \left(QU\right) \Sigma V^{T}
            \end{equation}
            Thus the $U$ matrix would be changed because of $Q$, however
            because $Q$ is orthogonal and $U$ is orthogonal then $QU$ would
            also be orthogonal.
        \end{mdframed}
\end{enumerate}
