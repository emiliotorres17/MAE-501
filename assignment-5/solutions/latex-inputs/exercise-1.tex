\section{Exercise 1}
\subsection{Part a}

Describe clearly but briefly the meaning of each term below, including
advantages and disadvantages, number of operations, etc.,  in your
\emph{\textbf{own}} words (not \emph{\textbf{anyone}} else's words, so not copied
from any book, Wikipedia, internet, etc.,) and without using
\emph{\textbf{any}} equations \emph{\textbf{whatsoever}}.

\begin{enumerate}[label=\arabic*.]
    \item Eigenvalues and eigenvectors
    \item Digitalization 
    \item Singular value decomposition
    \item Algebraic and geometric multiplicity
    \item Positive definite matrix   
\end{enumerate}

\subsection{Part b}

\begin{enumerate}[label=\arabic*.]
    \item Find the eigenvalues and eigenvectors of $A$, $A^{2}$, $A^{-1}$,
        and $A+4I$.
        \begin{equation}
            A =
            \begin{bmatrix}
                2  & -1 \\
                -1 & 2  \\
            \end{bmatrix}
        \end{equation}
        Check that the trace is $\lambda_{1} + \lambda_{2}$ and the
        determinant is $\lambda_{1} \lambda_{2}$ for $A$ and $A^{2}$.
        \begin{mdframed}[style=MyFrame]
            We can easily find the eigenvalues using $\deter(A-\lambda I) =
            0$. Starting by finding the eigenvalues of $A$ we get the
            following,
            \begin{equation}
                \deter(A-\lambda I) = 
                \deter
                \begin{bmatrix}
                    2-\lambda           &   -1              \\
                    -1                  &   2-\lambda
                \end{bmatrix}
                = 0
            \end{equation}
            This gives the following characteristic polynomial,
            \begin{equation}
                \lambda^{2}-4\lambda+3 = (\lambda -3)(\lambda -1)=0
            \end{equation}
            Therefore  $\lambda_{1} =3$ and $\lambda_{2} = 1$. Applying the
            same methodology to $A^{2}$ we get,
            \begin{equation}
                \deter \left(A^{2} -\lambda I \right) = 
                \deter
                \begin{bmatrix}
                    5-\lambda           &   -4             \\
                    -4                  &   5-\lambda
                \end{bmatrix}
                = 0
            \end{equation}
            This gives,
            \begin{equation}
                \left(5-\lambda\right)\left(5-\lambda\right)-16 = 0 
            \end{equation}
            Simplifying we get,
            \begin{equation}
                \lambda^{2} - 10\lambda + 9 = 
                \left(\lambda -9\right)\left(\lambda - 1\right) = 0
            \end{equation}
            Therefore $\lambda_{1} = 9$ and  $\lambda_{2} = 5$. Clearly we
            can see that the eigenvalues of $A^{2}$ are the square of the
            eigenvalues of $A$.  Moreover, applying the same logic as above
            we get the following characteristic equation for $A^{-1}$,
            \begin{equation}
                \lambda^{2} -\frac{4}{3} \lambda + \frac{1}{3} =
                \left(\lambda -1\right)\left(\lambda - \frac{1}{3}\right)
                = 0
            \end{equation}
            Thus, $\lambda_{1} = 1$ and $\lambda_{2} = \frac{1}{3}$.
            Lastly, it can be shown that the eigenvalues for $A+4I$ are
            $\lambda_{1} = 7$ and $\lambda_{2} = 3$. 
        \end{mdframed}
    \item Find the eigenvalues of $A$, $B$,$AB$, and $BA$:
        \begin{equation}
            A = 
            \begin{bmatrix}
                1  & 0 \\
                1  & 1 \\
            \end{bmatrix}
        \end{equation}
        and 
        \begin{equation}
            B = 
            \begin{bmatrix}
                1   & 2 \\
                0  & 1  \\
            \end{bmatrix}
        \end{equation}
        \begin{enumerate}[label=(\alph*)]
            \item Are the eigenvalues of $AB$ equal to eigenvalues of
                $A$ times the eigenvalues of $B$?
            \item Are the eigenvalues of $AB$ equal to the eigenvalues
                of $BA$?
        \end{enumerate}
        \begin{mdframed}[style=MyFrame]
            Again we can apply $\deter(A-\lambda I) = 0$ to find the
            eigenvalues for the above matrices. Starting with $A$ we
            get the following,
            \begin{equation}
                \deter \left(A\right) =
                \left(1-\lambda\right)\left(1-\lambda\right) = 0
            \end{equation}
            Therefore, $\lambda_{1} = \lambda_{2} =1$. Furthermore, for
            $B$ we get the following characteristic polynomial, 
            \begin{equation}
                \deter \left(B\right) = 
                \left(1-\lambda\right)\left(1-\lambda\right) = 0
            \end{equation}
            This gives $\lambda_{1} = \lambda_{2} = 1$ which we clearly
            are the same as the eigenvalues of $A$. 
            \\
            \\
            Next, we can evaluate both $AB$ and $BA$, starting with the
            eigenvalues of $AB$,
            \begin{equation}
                \deter(AB-\lambda I ) = 
                \left(1-\lambda\right)\left(3-\lambda\right)-2= 0
            \end{equation}
            Thus for $AB$ we get $\lambda_{1} = 2+\sqrt{3}$ and
            $\lambda_{2} =  2-\sqrt{3}$. Moreover, we can find the
            eigenvalues of $BA$ which gives,
            \begin{equation}
                \deter(BA-\lambda I ) = 
                \left(1-\lambda\right) \left(3-\lambda\right)-2= 0
            \end{equation}
            Clearly we can see that the characteristic polynomial for
            $BA$ and $AB$ are the same and  therefore they produce the
            same eigenvalues. Lastly, we can clearly see that the
            eigenvalues of $A$ times the eigenvalues of $B$ are not
            equal to the eigenvalues of $AB$. 
        \end{mdframed}

    \item  A $3 \times 3$ matrix $B$ is known to have eigenvalues 0, 1,
        2. This information is enough to find three of these (give the
        answer when possible):
        \begin{enumerate}[label=(\alph*)]
            \item The rank of $B$
                \begin{mdframed}[style=MyFrame]
                    The rank of any square matrix equal the number of
                    nonzero eigenvalues, therefore the rank of
                    matrix $B$ is 2.
                \end{mdframed}
            \item The determinant of $B^{T}B$
                \begin{mdframed}[style=MyFrame]
                    By factoring the determinant to both terms and
                    simplifying gives, 
                    \begin{equation}
                        \deter\left(B^{T}B\right)
                        = 
                        \deter\left(B^{T}\right)
                        \deter\left(B\right)
                        = 
                        \deter\left(B\right)^{2}
                        =
                        1 \cdot 2 \cdot 0
                        = 0
                    \end{equation}
                \end{mdframed}
            \item The eigenvalues of $B^{T}B$
                \begin{mdframed}[style=MyFrame]
                    Not enough information to obtain the eigenvalues
                    of $B^{T}B$.
                \end{mdframed}
            \item The eigenvalues of $(B^{2}+I)^{-1}$
                \begin{mdframed}[style=MyFrame]
                    From question 1 we can see that the eigenvalues
                    of $B^{2}$ are 0, 1, and 4. Additionally, we know
                    that the eigenvalues of $B^{2}+I$ are 1, 2, and
                    5. Thus the eigenvalues of $\left(B^{2} + I
                    \right)^{-1}$ must be $1$, $1/2$, and $1/5$. 
                \end{mdframed}
        \end{enumerate}
    \item Find the eigenvalues of $A$, $B$, and $C$
        \begin{equation}
            A =
            \begin{bmatrix}
                1 & 2 & 3 \\ 
                0 & 4 & 5 \\
                0 & 0 & 6
            \end{bmatrix}
        \end{equation}
        \begin{mdframed}[style=MyFrame]
            Staring with the definition of eigenvalues
            \begin{equation}
                \deter(A-\lambda I) =
                \deter
                \begin{bmatrix}
                    1-\lambda     &   2           &   3           \\ 
                    0             &   4-\lambda   &   5           \\
                    0             &   0           & 6-\lambda
                \end{bmatrix}
            \end{equation}
            This gives the following characteristic polynomial
            \begin{equation}
                \lambda^{3}-11\lambda^{2}+34\lambda+24
                =
                \left(\lambda-6\right)
                \left(\lambda-4\right)
                \left(\lambda-1\right)
                = 0
            \end{equation}
            Thus $\lambda_{1} = 6$, $\lambda_{2} = 4$, and $\lambda_{3}
            =1$. 
        \end{mdframed}
        \begin{equation}
            B =
            \begin{bmatrix}
                0 & 0 & 1 \\ 
                0 & 2 & 0 \\
                3 & 0 & 0
            \end{bmatrix}
        \end{equation}
        \begin{mdframed}[style=MyFrame]
            Using the same methodology as above gives,
            \begin{equation}
                \deter \left(B - \lambda I\right) 
                =
                \deter
                \begin{bmatrix}
                    0-\lambda   &   0           & 1             \\ 
                    0           &   2-\lambda   & 0             \\
                    3           &   0           & 0-\lambda 
                \end{bmatrix}
            \end{equation}
            This gives the following characteristic polynoimial,
            \begin{equation}
                \left(\lambda^{2}-3\right)\left(\lambda-2\right) = 0
            \end{equation}
            Thus $\lambda_{1} =  \sqrt{3}$, $\lambda_{2} = -\sqrt{3}$, and
            $\lambda_{3} =2 $.
        \end{mdframed}
            \begin{equation}
                C =
                \begin{bmatrix}
                    2 & 2 & 2 \\ 
                    2 & 2 & 2 \\
                    2 & 2 & 2
                \end{bmatrix}
            \end{equation}
            \begin{mdframed}[style=MyFrame]
                Again we get,
                \begin{equation}
                    \deter\left(C-\lambda I\right)
                    =
                    \deter
                    \begin{bmatrix}
                        2-\lambda       &       2           &   2           \\ 
                        2               &       2-\lambda   &   2           \\
                        2               &       2           &   2-\lambda
                    \end{bmatrix}
                \end{equation}
                This gives,
                \begin{equation}
                    \lambda^{2}\left(\lambda-6\right)=0
                \end{equation}
                Thus the eignevalues of $C$ are $\lambda_{1} = 6$ and
                $\lambda_{2} = \lambda_{3} = 0$.
            \end{mdframed}
    \end{enumerate}
