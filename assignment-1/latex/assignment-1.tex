\documentclass[12pt]{article}
%-------------------------------------------------------------------------%
% Preamble                                                                %
%-------------------------------------------------------------------------%
\newcommand{\pdftitle}{assignment-1}
\input{preamble}
\fancyhead[L]{MAE 501 Assignment 1}
\fancyhead[C]{}
\fancyhead[R]{E. Torres}
\fancyfoot[L]{\today}
\fancyfoot[C]{}
\fancyfoot[R]{\thepage}
%-------------------------------------------------------------------------%
% Document                                                                %
%-------------------------------------------------------------------------%
\begin{document}
\section{Exercise 1}
Solve the system of equations given in Eq~(\ref{eq:symbolic-system}) using
elimination, clearly showing the updated matrix after every row operation. 
\begin{equation}
    \begin{aligned}
        a_{11} x_{1} + a_{12} x_{2} + a_{13} x_{3} & = b_{1}    \\ 
        a_{21} x_{1} + a_{22} x_{2} + a_{23} x_{3} & = b_{2}    \\
        a_{31} x_{1} + a_{32} x_{2} + a_{33} x_{3} & = b_{3} 
    \end{aligned}
    \label{eq:symbolic-system}
\end{equation}
which gives
\begin{equation}
    \begin{pmatrix}
        a_{11}      &   a_{12}      &   a_{13}              \\
        a_{21}      &   a_{22}      &   a_{23}              \\
        a_{31}      &   a_{32}      &   a_{33}              \\
    \end{pmatrix}
    \begin{pmatrix}
        x_{1}       \\
        x_{2}       \\
        x_{3}       \\
    \end{pmatrix}
    =
    \begin{pmatrix}
        b_{1}       \\
        b_{2}       \\
        b_{3}       \\
    \end{pmatrix}
\end{equation}

\section{Exercise 2}
Solve the following system of equations, showing all relevant steps and
stating the method used to solve the system (e.g., $LU$ factorization,
direct numerical solution, matrix elimination, etc.,).  Furthermore, if a
system has no solution state it has no solution and briefly describe why it
has no solution. 
\begin{equation}
    \begin{pmatrix}
        1       &   3       &   1   \\
        4       &   0       &   2   \\
        1       &   3       &   4   \\
    \end{pmatrix}
    \begin{pmatrix}
        x       \\
        y       \\
        z       \\
    \end{pmatrix}
    \begin{pmatrix}
        5       \\
        6       \\
        3       \\
    \end{pmatrix}
\end{equation}

\begin{equation}
    \begin{pmatrix}
        1       &   3       &   1   \\
        4       &   1       &   2   \\
        0       &   0       &   4   \\
    \end{pmatrix}
    \begin{pmatrix}
        x       \\
        y       \\
        z       \\
    \end{pmatrix}
    \begin{pmatrix}
        1       \\
        2       \\
        1       \\
    \end{pmatrix}
\end{equation}

  
\begin{equation}
    \begin{pmatrix}
        1       &   3       &   1   \\
        4       &   1       &   2   \\
        0       &   0       &   4   \\
    \end{pmatrix}
    \begin{pmatrix}
        x       \\
        y       \\
        z       \\
    \end{pmatrix}
    \begin{pmatrix}
        1       \\
        2       \\
        1       \\
    \end{pmatrix}
\end{equation}

\begin{equation}
    \begin{pmatrix}
        0       &   1       &   2   &   0   &   0   \\
        5       &   0       &   2   &   0   &   0   \\
        0       &   0       &   0   &   1   &   1   \\
        0       &   0       &   0   &   0   &   1   \\
        0       &   0       &   1   &   0   &   3   \\
    \end{pmatrix}
    \begin{pmatrix}
        x_{1}   \\
        x_{2}   \\
        x_{3}   \\
        x_{4}   \\
        x_{5}   \\
    \end{pmatrix}
    \begin{pmatrix}
        1       \\
        2       \\
        3       \\
        4       \\
        5       \\
    \end{pmatrix}
\end{equation}
\newpage
\section{Exercise 3}
Describe clearly but briefly the meaning of each term below, including
advantages and disadvantages, number of operations, etc.,  in your
\underline{own} words (not \underline{anyone} else's words, so not copied
from any book, Wikipedia, internet, etc.,) and without using
\underline{any} equations \underline{whatsoever}.

\begin{enumerate}[label=(\alph*)]
    \item Direct Elimination 
    \item $LU$ Factorization
    \item Inverse Matrices 
    \item Symmetric Matrices  
    \item Transpose
    \item Permutation 
    \item Inner product
    \item Singular Matrices
\end{enumerate}

\section{Exercise 4}
Solve the following system  and verify the solution visually by plotting
the two systems using \underline{Python} and annotating the interception
point. 
\begin{equation}
    \begin{aligned}
        2x  + 3y    = 4 \\
        x   + 2y    = 5 
    \end{aligned}
\end{equation}
\newpage
\section{Exercise 5}
Use
\begin{equation}
    A   = 
    \begin{pmatrix}
        a_{11}  &   a_{12}  &   a_{13}  \\
        a_{21}  &   a_{22}  &   a_{23}  \\
        a_{31}  &   a_{32}  &   a_{33}  \\
    \end{pmatrix}
\end{equation}

\begin{equation}
    B   = 
    \begin{pmatrix}
        b_{11}  &   b_{12}  &   b_{13}  \\
        b_{21}  &   b_{22}  &   b_{23}  \\
        b_{31}  &   b_{32}  &   b_{33}  \\
    \end{pmatrix}
\end{equation}
and 
\begin{equation}
    C   = 
    \begin{pmatrix}
        c_{11}  &   c_{12}  &   c_{13}  \\
        c_{12}  &   c_{22}  &   c_{23}  \\
        c_{13}  &   c_{23}  &   c_{33}  \\
    \end{pmatrix}
\end{equation}

to verify the following
\begin{enumerate}[label=(\alph*)]
    \item $\left(AB\right)^{T} = B^{T} A^{T}$ 
    \item $\left(A^{-1}\right)^{T} = \left(A^{T}\right)^{-1}$  
    \item $C^{T} = C$
\end{enumerate}
    
\end{document}
