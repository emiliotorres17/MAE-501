\documentclass[12pt]{article}
%-------------------------------------------------------------------------%
% Preamble                                                                %
%-------------------------------------------------------------------------%
\newcommand{\pdftitle}{assignment-1}
%=========================================================================%
%latex packages                                                           %
%=========================================================================%

%\usepackage[a4paper, bindingoffset = 0.2in,
%            left = 1in, right = 1in, top = 1in,
%            bottom = 1in, footskip = 0.25in]{geometry}     %setting the paper margin
                                        %used to break the citations for ling citations
\usepackage{breakcites}
\usepackage{caption}
\usepackage{subcaption}
\usepackage[margin = 0.9in]{geometry}
\usepackage[euler]{textgreek}           %used for the text greek symbols in the names
\usepackage{physics}                    %used for the equations
\usepackage{array}                      %used for setting up the tables
\usepackage{enumitem}
%\usepackage{float}                      %places the figures and tables
\usepackage{amssymb}
\usepackage{lastpage}                   %labels the last page
\usepackage{graphicx}                   %making the figures
\usepackage{color}                      %full color palette
\usepackage{longtable}                  %makes the table
\usepackage{fancyhdr}                   %setting the paper style
\usepackage{siunitx}                    %used for the units
\usepackage{sectsty}                    %used for the section style
\usepackage{blindtext}                  %used to handle the blind text
\usepackage{amsmath}                    %math package
\usepackage{ifxetex}                    %used for the MD5SUM
\usepackage[flushleft]{threeparttable}  %making the footnotes in tables
\usepackage{threeparttablex}            %second part for the table
\usepackage{pdftexcmds}                 %used to perform pdf commands
\usepackage{lscape}                     %used to make the page landscape
\usepackage{refcount}                   %used to reference numbers
\usepackage{listings}                   %provide source code listing
\usepackage{tikz}                       %for a black box on title page
\usepackage{pdfpages}                   %for pdf pages to be added
\usepackage[printwatermark]{xwatermark} %used for water mark
\usepackage{xcolor}                     %water mark color
\usepackage{booktabs}
\usepackage{chngcntr}
\usepackage{floatrow}                   %centering all figures
\usepackage[section]{placeins}          %controls float placement to ensure figures are in proper
\usepackage[none]{hyphenat}             % prevent hyphenation
\usepackage{scrextend}
\usepackage{xr-hyper}                   %allow references between documents
\usepackage{hyperref}                   %highlights references
\usepackage[all]{hypcap}                %for going to the top of an image when a figure ref is clicked
%=========================================================================%
%hyperlink setup                                                          %
%=========================================================================%
\definecolor{codegreen}{rgb}{0,0.6,0}
\definecolor{codegray}{rgb}{0.5,0.5,0.5}
\definecolor{codepurple}{rgb}{0.58,0,0.82}
\definecolor{backcolour}{rgb}{0.95,0.95,0.92}
\definecolor{citeblue}{RGB}{015, 055, 250}
\definecolor{gblue}{RGB}{055, 055, 150}
\definecolor{phxorange}{RGB}{229, 96, 32}
\definecolor{dodgerblue}{RGB}{0, 90, 156}
\definecolor{eggshell}{RGB}{240, 234, 214}
\definecolor{pythonblue}{RGB}{24, 23, 162}

\hypersetup{%color all links blue
                plainpages=false,
                pdftitle={\pdftitle},
                pdfauthor={Emilio Torres},
                colorlinks=true,
                citecolor=citeblue,
                filecolor=gblue,
                linkcolor=gblue,
                menucolor=gblue,
                runcolor=gblue,
                urlcolor=gblue,
}
%=========================================================================%
%Temperature                                                              %
%=========================================================================%
\newcommand{\mDC}{^{\circ}\text{C}}
\newcommand{\DC}{$^{\circ}\text{C}$}

\newcommand{\DF}{$^{\circ}\text{F}$}
\newcommand{\mDF}{^{\circ}\text{F}}

\newcommand{\mDR}{^{\circ}\text{R}}
\newcommand{\DR}{$^{\circ}\text{R}$}

\newcommand{\mDK}{\text{K}}
\newcommand{\DK}{$\text{K}$}
%=========================================================================%
%Multiplier                                                               %
%=========================================================================%
\newcommand{\multi}{ \times }
%=========================================================================%
%unit number setup                                                        %
%=========================================================================%
\sisetup
{
    exponent-product        =   \multi,
    exponent-base           =   10,
    group-digits            =   true,
    group-separator         =   {},
    output-decimal-marker   =   {.},
    retain-explicit-plus,
}
%=========================================================================%
%Percent                                                              %
%=========================================================================%
\newcommand{\percent}{\%{\;}}
%=========================================================================%
%listing setup                                                            %
%=========================================================================%
\lstset{language=python,
    basicstyle=\footnotesize \ttfamily,
    commentstyle=\color{codegreen},
    keywordstyle=\color{magenta},
    numberstyle=\tiny\color{codegray},
    stringstyle=\color{codepurple},
    breaklines=true,
    morekeywords={matlab2tikz},
    morekeywords=[2]{1}, keywordstyle=[2]{\color{black}},
    identifierstyle=\color{black},
    showstringspaces=false,%without this there will be a symbol in the places where there is a space
    numbers=left,
    numberstyle={\tiny \color{black}},% size of the numbers
    numbersep=9pt, % this defines how far the numbers are from the text
    emph=[1]{for,end,break},emphstyle=[1]\color{red}, %some words to emphasise
}
%=========================================================================%
% Page Setup                                                              %
%=========================================================================%
%-------------------------------------------------------------------------%
%Header                                                                   %
%-------------------------------------------------------------------------%
\pagestyle{fancy}
\renewcommand{\headrulewidth}{0.0pt}
\fancyhead[L]{}
\fancyhead[C]{}
\fancyhead[R]{}
%-------------------------------------------------------------------------%
%Footer                                                                   %
%-------------------------------------------------------------------------%
\renewcommand{\footrulewidth}{0.0pt}
\fancyfoot[L]{}
\fancyfoot[R]{\thepage}
\fancyfoot[C]{}
%-------------------------------------------------------------------------%
%Page Commands                                                            %
%-------------------------------------------------------------------------%
\setcounter{secnumdepth}{3}
\setlength{\parindent}{0cm}
\setlength{\parskip}{10pt}
%=========================================================================%
% User Defined Functions                                                  %
%=========================================================================%
%-------------------------------------------------------------------------%
% Table spacing functions                                                 %
%-------------------------------------------------------------------------%
%provide a centered horizontal & centered vertical wrapped table column
\newcolumntype{A}[1]{>{\centering\arraybackslash}m{#1}}
%provide a centered horizontal & bottom vertical wrapped table column
\newcolumntype{B}[1]{>{\centering\arraybackslash}p{#1}}
%provide a left horizontal & bottom vertical wrapped table column
\newcolumntype{D}[1]{>{\raggedright\arraybackslash}p{#1}}
%=========================================================================%
% User Defined Functions                                                  %
%=========================================================================%
%-------------------------------------------------------------------------%
% Elements and Math                                                       %
%-------------------------------------------------------------------------%
\newcommand{\element}[2]{$\text{#1}_{#2}$}
\newcommand{\melement}[2]{\text{#1}_{#2}}
%-------------------------------------------------------------------------%
% Isotopes                                                                %
%-------------------------------------------------------------------------%
\newcommand{\isotope}[2]{${}^{#2} \text{#1}$}
\newcommand{\misotope}[2]{{}^{#2} \text{#1}}
%-------------------------------------------------------------------------%
% Uranium Dioxide                                                         %
%-------------------------------------------------------------------------%
\newcommand{\UO}{$\text{U}\text{O}_{2}$}
\newcommand{\mUO}{\text{U}\text{O}_{2}}
%-------------------------------------------------------------------------%
% Di-gadium Tri-oxide                                                        %
%-------------------------------------------------------------------------%
\newcommand{\GdO}{$\text{Gd}_{2}\text{O}_{3}$}
\newcommand{\mGdO}{\text{Gd}_{2}\text{O}_{3}}
%-------------------------------------------------------------------------%
% Plutonium Dioxide                                                       %
%-------------------------------------------------------------------------%
\newcommand{\PuO}{$\text{PuO}_{2}$}
\newcommand{\mPuO}{\text{PuO}_{2}}
%-------------------------------------------------------------------------%
% Misc. Command                                                           %
%-------------------------------------------------------------------------%
\newcommand{\Temp}[2]{$#1\text{#2}$}
\newcommand{\der}{$d \varepsilon_{r}^{p}$}
\newcommand{\detheta}{$d \varepsilon_{\theta}^{p}$}
\newcommand{\dez}{$d \varepsilon_{z}^{p}$}
\newcommand{\de}[1]{$d \varepsilon_{#1}^{p}$}
\newcommand{\er}{$\varepsilon_{r}^{p}$}
\newcommand{\etheta}{$\varepsilon_{\theta}^{p}$}
\newcommand{\ez}{$\varepsilon_{z}^{p}$}
\newcommand{\Volume}[2]{$#1\;\text{#2}^{3}\;$}
%-------------------------------------------------------------------------%
% slash indent command                                                    %
%-------------------------------------------------------------------------%
\newcommand{\ind}[1]{\vspace{0.25cm}\begin{addmargin}[0.5cm]{0em}#1\end{addmargin}\vspace{0.25cm}}
%-------------------------------------------------------------------------%
% French spacing                                                          %
%-------------------------------------------------------------------------%
\frenchspacing
%-------------------------------------------------------------------------%
% figure size                                                             %
%-------------------------------------------------------------------------%
\newcommand{\figheight}{0.4}
\newcommand{\figheighttwo}[1]{#1}
%-------------------------------------------------------------------------%
% figure size 2                                                           %
%-------------------------------------------------------------------------%
\newcommand{\test}{height=0.4\\textheight}
%-------------------------------------------------------------------------%
% SI-Unitx Commands                                                       %
%-------------------------------------------------------------------------%
\newcommand{\SInum}[2]{\SI{#1}{\left[#2\right]}}
\newcommand{\sinum}[1]{\si{\left[#1\right]}}
%-------------------------------------------------------------------------%
% Closing inline references (i.e., (), [], {})                            %
%-------------------------------------------------------------------------%
\newcommand{\refclosure}[1]{(#1)}
%-------------------------------------------------------------------------%
% Trademark Command                                                       %
%-------------------------------------------------------------------------%
\newcommand{\TM}{${}^{\text{TM}}$ }
\newcommand{\captiontext}{TEST}
%-------------------------------------------------------------------------%
% iteration function                                                      %
%-------------------------------------------------------------------------%
\newcommand{\iteration}[1]{#1-th}
%-------------------------------------------------------------------------%
% notes function                                                          %
%-------------------------------------------------------------------------%
\newcommand{\notes}[1]{\textit{#1}}
%-------------------------------------------------------------------------%
% Density function                                                        %
%-------------------------------------------------------------------------%
\newcommand{\Gden}{G_{den}}
%-------------------------------------------------------------------------%
% Reference place holders for in  text figures, tables , etc.             %
%-------------------------------------------------------------------------%
\newcommand{\refplaceholder}[1]{{\color{red}{reference #1 goes here}}}
%-------------------------------------------------------------------------%
% Equation labelling                                                      %
%-------------------------------------------------------------------------%
\newcommand{\eqlabel}[1]{Eq.~\ref{#1}}
%-------------------------------------------------------------------------%
% Making dashline                                                         %
%-------------------------------------------------------------------------%
\newcommand{\dashline}[1]
    {
    \raisebox{1mm}{
        \textcolor{#1}{\rule{0.25cm}{0.5mm}}
        \hspace{0.025cm}
        \textcolor{#1}{\rule{0.25cm}{0.5mm}}
        \hspace{0.025cm}
        \textcolor{#1}{\rule{0.25cm}{0.5mm}}
        }
    }
%-------------------------------------------------------------------------%
% Color box                                                               %
%-------------------------------------------------------------------------%
\newcommand*{\colorboxed}{}
\def\colorboxed#1#{%
    \colorboxedAux{#1}%
}
\newcommand*{\colorboxedAux}[3]{%
 % #1: optional argument for color model
 % #2: color specification
 % #3: formula
 \begingroup
   \colorlet{cb@saved}{.}%
   \color#1{#2}%
   \boxed{%
     \color{cb@saved}%
     #3%
   }%
 \endgroup
}
%-------------------------------------------------------------------------%
% Expanding matrices                                                      %
%-------------------------------------------------------------------------%
\makeatletter
\renewcommand*\env@matrix[1][\arraystretch]{%
  \edef\arraystretch{#1}%
  \hskip -\arraycolsep
  \let\@ifnextchar\new@ifnextchar
  \array{*\c@MaxMatrixCols c}}
\makeatother

\fancyhead[L]{MAE 501 Assignment 1}
\fancyhead[C]{}
\fancyhead[R]{E. Torres}
\fancyfoot[L]{\today}
\fancyfoot[C]{}
\fancyfoot[R]{\thepage}
%-------------------------------------------------------------------------%
% Document                                                                %
%-------------------------------------------------------------------------%
\begin{document}
\section{Exercise 1}
Solve the system of equations given in Eq~(\ref{eq:symbolic-system}) using
elimination, clearly showing the updated matrix after every row operation. 
\begin{equation}
    \begin{aligned}
        a_{11} x_{1} + a_{12} x_{2} + a_{13} x_{3} & = b_{1}    \\ 
        a_{21} x_{1} + a_{22} x_{2} + a_{23} x_{3} & = b_{2}    \\
        a_{31} x_{1} + a_{32} x_{2} + a_{33} x_{3} & = b_{3} 
    \end{aligned}
    \label{eq:symbolic-system}
\end{equation}
which gives
\begin{equation}
    \begin{bmatrix}
        a_{11}      &   a_{12}      &   a_{13}              \\
        a_{21}      &   a_{22}      &   a_{23}              \\
        a_{31}      &   a_{32}      &   a_{33}              \\
    \end{bmatrix}
    \begin{bmatrix}
        x_{1}       \\
        x_{2}       \\
        x_{3}       \\
    \end{bmatrix}
    =
    \begin{bmatrix}
        b_{1}       \\
        b_{2}       \\
        b_{3}       \\
    \end{bmatrix}
\end{equation}

\section{Exercise 2}
Solve the following system of equations, showing all relevant steps and
stating the method used to solve the system (e.g., $LU$ factorization,
direct numerical solution, matrix elimination, etc.,).  Furthermore, if a
system has no solution state it has no solution and briefly describe why it
has no solution. 
\begin{equation}
    \begin{bmatrix}
        1       &   3       &   1   \\
        4       &   0       &   2   \\
        1       &   3       &   4   \\
    \end{bmatrix}
    \begin{bmatrix}
        x       \\
        y       \\
        z       \\
    \end{bmatrix}
    \begin{bmatrix}
        5       \\
        6       \\
        3       \\
    \end{bmatrix}
\end{equation}

\begin{equation}
    \begin{bmatrix}
        1       &   3       &   1   \\
        4       &   1       &   2   \\
        0       &   0       &   4   \\
    \end{bmatrix}
    \begin{bmatrix}
        x       \\
        y       \\
        z       \\
    \end{bmatrix}
    \begin{bmatrix}
        1       \\
        2       \\
        1       \\
    \end{bmatrix}
\end{equation}

  
\begin{equation}
    \begin{bmatrix}
        1       &   3       &   1   \\
        4       &   1       &   2   \\
        0       &   0       &   4   \\
    \end{bmatrix}
    \begin{bmatrix}
        x       \\
        y       \\
        z       \\
    \end{bmatrix}
    \begin{bmatrix}
        1       \\
        2       \\
        1       \\
    \end{bmatrix}
\end{equation}

\begin{equation}
    \begin{bmatrix}
        0       &   1       &   2   &   0   &   0   \\
        5       &   0       &   2   &   0   &   0   \\
        0       &   0       &   0   &   1   &   1   \\
        0       &   0       &   0   &   0   &   1   \\
        0       &   0       &   1   &   0   &   3   \\
    \end{bmatrix}
    \begin{bmatrix}
        x_{1}   \\
        x_{2}   \\
        x_{3}   \\
        x_{4}   \\
        x_{5}   \\
    \end{bmatrix}
    \begin{bmatrix}
        1       \\
        2       \\
        3       \\
        4       \\
        5       \\
    \end{bmatrix}
\end{equation}
\newpage
\section{Exercise 3}
Describe clearly but briefly the meaning of each term below, including
advantages and disadvantages, number of operations, etc.,  in your
\underline{own} words (not \underline{anyone} else's words, so not copied
from any book, Wikipedia, internet, etc.,) and without using
\underline{any} equations \underline{whatsoever}.

\begin{enumerate}[label=(\alph*)]
    \item Direct Elimination 
    \item $LU$ Factorization
    \item Inverse Matrices 
    \item Symmetric Matrices  
    \item Transpose
    \item Permutation 
    \item Inner product
    \item Singular Matrices
\end{enumerate}

\section{Exercise 4}
Solve the following system  and verify the solution visually by plotting
the two systems using \underline{Python} and annotating the interception
point. 
\begin{equation}
    \begin{aligned}
        2x  + 3y    = 4 \\
        x   + 2y    = 5 
    \end{aligned}
\end{equation}
\newpage
\section{Exercise 5}
Use
\begin{equation}
    A   = 
    \begin{bmatrix}
        a_{11}  &   a_{12}  &   a_{13}  \\
        a_{21}  &   a_{22}  &   a_{23}  \\
        a_{31}  &   a_{32}  &   a_{33}  \\
    \end{bmatrix}
\end{equation}

\begin{equation}
    B   = 
    \begin{bmatrix}
        b_{11}  &   b_{12}  &   b_{13}  \\
        b_{21}  &   b_{22}  &   b_{23}  \\
        b_{31}  &   b_{32}  &   b_{33}  \\
    \end{bmatrix}
\end{equation}
and 
\begin{equation}
    C   = 
    \begin{bmatrix}
        c_{11}  &   c_{12}  &   c_{13}  \\
        c_{12}  &   c_{22}  &   c_{23}  \\
        c_{13}  &   c_{23}  &   c_{33}  \\
    \end{bmatrix}
\end{equation}

to verify the following
\begin{enumerate}[label=(\alph*)]
    \item $\left(AB\right)^{T} = B^{T} A^{T}$ 
    \item $\left(A^{-1}\right)^{T} = \left(A^{T}\right)^{-1}$  
    \item $C^{T} = C$
\end{enumerate}

\section{Exercise 6}
Find the $LU$ and $LDU$ factorizations by hand for the following matrices.

\begin{equation}
    A = 
    \begin{bmatrix}
        1       &       3       &   2   \\
        3       &       5       &   4   \\
        2       &       4       &   6   \\
    \end{bmatrix}
\end{equation}

\begin{equation}
    B = 
    \begin{bmatrix}
        1       &       0       &   1   \\
        2       &       2       &   2   \\
        3       &       4       &   5   \\
    \end{bmatrix}
\end{equation}

\begin{equation}
    C = 
    \begin{bmatrix}
        2       &       1       &   3   \\
        3       &       2       &   3   \\
        3       &       1       &   0   \\
    \end{bmatrix}
\end{equation}

\section{Exercise 7}
Using Python generate a $LU$ factorization tool (subroutine) that takes
in a $N \times N$ matrix and outputs both $L$ and $U$ factors. Next, use
your factorization tool to compute the $LU$ factors for a matrix with 
$N=10, 100, 200, 300, 400, 500, 1000$ and plot $N$ versus time it takes to
compute the factorization. Use the pseudo code provided to learn see how to
time different blocks of code and generate random matrices. 

\section{Exercise 8}
Find the inverse by hand for the following matrices. If a matrix does not
exist briefly state why it cannot be inverted.

\begin{equation}
    A = 
    \begin{bmatrix}
        1       &       0       &   1   \\
        2       &       1       &   3   \\
        0       &       0       &   1   \\
    \end{bmatrix}
\end{equation}

\begin{equation}
    B = 
    \begin{bmatrix}
        2       &       1       &   1   \\
        1       &       2       &   1   \\
        1       &       1       &   2   \\
    \end{bmatrix}
\end{equation}

\begin{equation}
    C = 
    \begin{bmatrix}
        2       &       -1      &   -1  \\
        -1      &       2       &   -1  \\
        -1      &       -1      &   2   \\
    \end{bmatrix}
\end{equation}

\section{Exercise 9}
Using Python generate a matrix inverse tool (subroutine) that takes in a $N
\times N$ matrix and outputs its inverse. Similarly to the previous
assignment, use your inverse tool to compute the $A^{-1}$ for a matrix
with $N=10, 100, 200, 300, 400, 500, 1000$ and plot $N$ versus time it
takes to compute the inverse.


    
\end{document}
