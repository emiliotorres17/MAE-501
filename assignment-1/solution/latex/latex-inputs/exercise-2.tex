\section{Exercise 2}
Solve the following system of equations, showing all relevant steps and
stating the method used to solve the system (e.g., $LU$ factorization,
direct numerical solution, matrix elimination, etc.,).  Furthermore, if a
system has no solution state it has no solution and briefly describe why it
has no solution. 
\begin{enumerate}[label=(\alph*)]
    \item The following system of equations can be solved using the
        inverse matrix approach, namely
        \begin{subequations}
            \begin{equation}
                \mathbf{x} = \mathbf{A}^{-1}\mathbf{b}
            \end{equation}
            therefore we need to obtain $A^{-1}$. We can determine the
            inverse of matrix $\mathbf{A}$ using the augmented matrix
            method. Where the matrix $\mathbf{A}$ is augmented with the
            identity matrix $\mathbf{I}$, and row operations are performed
            until the left hand side of the augmented matrix is the
            identity matrix. Starting with the augmented matrix
            \begin{equation}
                \begin{bmatrix}[ccc|ccc]
                    1   &   3   &   1   &   1   &   0   &   0   \\
                    4   &   0   &   2   &   0   &   1   &   0   \\
                    1   &   3   &   4   &   0   &   0   &   1   
                \end{bmatrix}
            \end{equation}
            Then we can take $4r_{1} - r_{2}$ to zero out the $a_{21}$
            term giving
            \begin{equation}
                \begin{bmatrix}[ccc|ccc]
                    1   &   3   &   1   &   1   &   0   &   0   \\
                    0   &   12  &   2   &   4   &   -1  &   0   \\
                    1   &   3   &   4   &   0   &   0   &   1   
                \end{bmatrix}
            \end{equation}
            Next, the $a_{31}$ is zeroed out by performing $r_{1} - r_{3}$
            giving
            \begin{equation}
                \begin{bmatrix}[ccc|ccc]
                    1   &   3   &   1   &   1   &   0   &   0   \\
                    0   &   12  &   2   &   4   &   -1  &   0   \\
                    0   &   0   &   -3  &   1   &   0   &   -1   
                \end{bmatrix}
            \end{equation}
            Now we continue zeroing out entries of $\mathbf{A}$ moving up
            the matrix starting with $a_{23}$ by performing $-\frac{2}{3}r_{3} -
            r_{2}$ giving
            \begin{equation}
                \begin{bmatrix}[ccc|ccc]
                    1   &   3   &   1   &   1               &   0   &   0               \\
                    0   &   -12 &   0   &  -14/3    &   1   &   2/3     \\
                    0   &   0   &   -3  &   1               &   0   &   -1   
                \end{bmatrix}
            \end{equation}
            Moving up the third column we zero out $a_{13}$ using
            $-\frac{1}{3}r_{3}-r_{1}$
            giving
            \begin{equation}
                \begin{bmatrix}[ccc|ccc]
                    -1  &   -3  &   0   &   -1/3    &   0   &   -1/3    \\
                    0   &   -12 &   0   &  -14/3    &   1   &   2/3     \\
                    0   &   0   &   -3  &   1               &   0   &   -1   
                \end{bmatrix}
            \end{equation}
            Next we can zero $a_{12}$ with $\frac{1}{4}r_{2}-r_{1}$ giving
            \begin{equation}
                \begin{bmatrix}[ccc|ccc]
                    1   &   0   &   0   &   1/6     &   1/4     &   -1/6    \\
                    0   &   -12 &   0   &  -14/3    &   1               &   2/3     \\
                    0   &   0   &   -3  &   1               &   0               &   -1   
                \end{bmatrix}
            \end{equation}
            Lastly we can multiply each pivot by its reciprocal, i.e.,
            $r_{2}=-\frac{r_{2}}{12}$ and $r_{3} = -\frac{r_{3}}{3}$, to
            finally get the inverse of matrix $\mathbf{A}$,
            \begin{equation}
                \begin{bmatrix}[ccc|ccc]
                    1   &   0   &   0   &   1/6     &   1/4     &   -1/6    \\
                    0   &   1   &   0   &   7/18    &   -1/12   &   -1/18    \\
                    0   &   0   &   1   &   -1/3    &   0       &   1/3   
                \end{bmatrix}
            \end{equation}
            Thus 
            \begin{empheq}[box=\widefbox]{equation}
                \mathbf{x} = \mathbf{A}^{-1}\mathbf{b} =
                    \begin{bmatrix}[ccc]
                        1/6     &   1/4     &   -1/6    \\  
                        7/18    &   -1/12   &   -1/18    \\ 
                        -1/3    &   0               &   1/3         
                    \end{bmatrix}
                    \begin{bmatrix}[c]
                        5   \\
                        6   \\
                        3
                    \end{bmatrix}
                    =
                    \begin{bmatrix}[c]
                        11/6      \\
                        23/18     \\
                        -2/3
                    \end{bmatrix}
            \end{empheq}
        \end{subequations}
    
    \item 
        \begin{mdframed}[style=MyFrame]
            From the system of equations
            \begin{equation}
                \begin{bmatrix}
                    1   & 4     &       2   \\
                    2   & 8     &       4   \\
                    1   & 5     &       4   
                \end{bmatrix}
                \begin{bmatrix}
                    x   \\
                    y   \\
                    z
                \end{bmatrix}
                \begin{bmatrix}
                    1   \\
                    2   \\
                    1
                \end{bmatrix}
            \end{equation}
            it is obvious that the second row is a linear
            combination of the first row, i.e. $r_{2}=2r_{1}$. Therefore the
            second equation in the system adds no new information, that we do
            not already have from the first row. Thus, we have a system with
            three unknowns, $x,y,z$, and only two equations, therefore the
            system does not have an unique solution.    
        \end{mdframed}
    
    \item The simplest way to solve the following system  
        \begin{subequations}
            \begin{equation}
                \begin{bmatrix}
                    1       &       3       &   1   \\
                    4       &       1       &   2   \\
                    0       &       0       &   4   \\
                \end{bmatrix}
                \begin{bmatrix}
                    x   \\
                    y   \\
                    z   
                \end{bmatrix}
                =
                \begin{bmatrix}
                    1   \\
                    2   \\
                    1   
                \end{bmatrix}
            \end{equation}
            is to use to use the first row to cancel out the first entry of
            the second row, i.e. $4r_{1}-r_{2}$, giving 
            \begin{equation}
                \begin{bmatrix}
                    1       &       3       &   1   \\
                    0       &       11      &   2   \\
                    0       &       0       &   4   \\
                \end{bmatrix}
                \begin{bmatrix}
                    x   \\
                    y   \\
                    z   
                \end{bmatrix}
                =
                \begin{bmatrix}
                    1   \\
                    2   \\
                    1   
                \end{bmatrix}
            \end{equation}
            and them backwards substituting. Thus,
            \begin{empheq}[box=\widefbox]{equation}
                \begin{bmatrix}
                    x   \\
                    y   \\
                    z   \\
                \end{bmatrix}        =
                \begin{bmatrix}
                    15/44       \\
                    3/22        \\
                    1/4         \\
                \end{bmatrix}
            \end{empheq}
        \end{subequations}

    \item The easiest way to solve this system of equations 
        \begin{subequations}
            \begin{equation}
                \begin{bmatrix}
                    0       &   1       &   2   &   0   &   0   \\
                    5       &   0       &   2   &   0   &   0   \\
                    0       &   0       &   0   &   1   &   1   \\
                    0       &   0       &   0   &   0   &   1   \\
                    0       &   0       &   1   &   0   &   3   \\
                \end{bmatrix}
                \begin{bmatrix}
                    x_{1}   \\
                    x_{2}   \\
                    x_{3}   \\
                    x_{4}   \\
                    x_{5}   \\
                \end{bmatrix}
                \begin{bmatrix}
                    1       \\
                    2       \\
                    3       \\
                    4       \\
                    5       \\
                \end{bmatrix}
            \end{equation}
        is to swap the rows so that all the pivots  
        are non-zero giving 
            \begin{equation}
                \begin{bmatrix}
                    5       &   0       &   2   &   0   &   0   \\
                    0       &   1       &   2   &   0   &   0   \\
                    0       &   0       &   1   &   0   &   3   \\
                    0       &   0       &   0   &   1   &   1   \\
                    0       &   0       &   0   &   0   &   1   \\
                \end{bmatrix}         =
                \begin{bmatrix}
                    x_{1}   \\
                    x_{2}   \\
                    x_{3}   \\
                    x_{4}   \\
                    x_{5}   \\
                \end{bmatrix}        =
                \begin{bmatrix}
                    2       \\
                    1       \\
                    5       \\
                    3       \\
                    4       \\
                \end{bmatrix}
            \end{equation}
        \end{subequations}
        Now the system of equations can be solved quite easily using
        backwards substitution. Thus, 
            \begin{empheq}[box=\widefbox]{equation}
                \begin{bmatrix}
                    x_{1}   \\
                    x_{2}   \\
                    x_{3}   \\
                    x_{4}   \\
                    x_{5}   \\
                \end{bmatrix}        =
                \begin{bmatrix}
                    16/5    \\
                    15      \\
                    -7      \\
                    -1      \\
                    4       
                \end{bmatrix}
            \end{empheq}

\end{enumerate}
