\section{Exercise 4}
Prove the following matrix properties:
\begin{enumerate}[label=(\alph*)]

    \item Prove the following:
        \begin{subequations} 
            \begin{equation}
                \left(AB\right)^{T} = B^{T}A^{T}
                \label{eq:ex-4-a}
            \end{equation}
            Start be defining two $N \times N$ matrices $\mathbf{A}$ and
            $\mathbf{B}$,
                \begin{equation}
                    \mathbf{A} = A_{ij} =
                    \begin{bmatrix}
                        a_{1,1}     &   a_{1,2}     & \cdots    & a_{1,n}    \\
                        a_{2,1}     &   a_{2,2}     & \cdots    & a_{2,n}    \\
                        \vdots      &  \vdots       & \ddots    & \vdots     \\
                        a_{n,1}     & a_{n_2}       & \cdots    & a_{n,n}  
                    \end{bmatrix}
                \end{equation}
                \begin{equation}
                    \mathbf{B} = B_{ij} =
                    \begin{bmatrix}
                        b_{1,1}     &   b_{1,2}     & \cdots    & b_{1,n}    \\
                        b_{2,1}     &   b_{2,2}     & \cdots    & b_{2,n}    \\
                        \vdots      &   \vdots      & \ddots    & \vdots     \\
                        b_{n,1}     &   b_{n,2}     & \cdots    & b_{n,n}  
                    \end{bmatrix}
                \end{equation}
                which gives the following for the LHS of Eq.~(\ref{eq:ex-4-a})
                \begin{equation}
                    AB  = \underbrace{\sum^{N}_{k=1}{A_{ik}B_{kj}}}_{\equiv C_{ij}}
                    =
                    \begin{bmatrix}
                        a_{1,1} b_{1,1} + a_{1,2} b_{2,1}   &   a_{1,1} b_{1,2} + a_{1,2} b_{2,2}   &   \cdots   \\  
                        a_{2,1} b_{1,1} + a_{2,2} b_{2,1}   &   a_{2,1} b_{1,2} + a_{2,2} b_{2,2}   &   \cdots   \\  
                        \vdots                              &   \ddots                              &   \vdots
                    \end{bmatrix}
                \end{equation}
                Therefore,
                \begin{equation}
                    C^{T} = 
                    \begin{bmatrix}
                        a_{1,1} b_{1,1} + a_{1,2} b_{2,1}   &   a_{2,1} b_{1,1} + a_{2,2} b_{2,1}   &   \cdots   \\  
                        a_{1,1} b_{1,2} + a_{1,2} b_{2,2}   &   a_{2,1} b_{1,2} + a_{2,2} b_{2,2}   &   \cdots   \\  
                        \vdots                              &   \ddots                              &   \vdots
                    \end{bmatrix}
                    \label{eq:C-transpose}
                \end{equation}
                Next we can look at the RHS of Eq.~(\ref{eq:ex-4-a}) where,
                \begin{equation}
                    A^{T} =
                    \begin{bmatrix}
                        a_{1,1}     &   a_{2,1}     & \cdots    & a_{n,1}    \\
                        a_{1,2}     &   a_{2,2}     & \cdots    & a_{n,2}    \\
                        \vdots      &  \vdots       & \ddots    & \vdots     \\
                        a_{1,n}     & a_{2,n}       & \cdots    & a_{n,n}  
                    \end{bmatrix}
                \end{equation}
                \begin{equation}
                    B^{T} =
                    \begin{bmatrix}
                        b_{1,1}     &   b_{2,1}     & \cdots    & b_{n,1}    \\
                        b_{1,2}     &   b_{2,2}     & \cdots    & b_{n,2}    \\
                        \vdots      &  \vdots       & \ddots    & \vdots     \\
                        b_{1,n}     & b_{2,n}       & \cdots    & b_{n,n}  
                    \end{bmatrix}
                \end{equation}
                Therefore,
                \begin{equation}
                    B^{T}A^{T}
                    =
                    \begin{bmatrix}
                        b_{1,1}a_{1,1} + b_{2,1}a_{1,2}     &   b_{1,1}a_{2,1} + b_{2,1}a_{2,2}     &   \cdots  \\
                        b_{1,2}a_{1,1} + b_{2,2}a_{1,2}     &   b_{1,2}a_{2,1} + b_{2,2}a_{2,2}     &   \cdots  \\
                        \vdots                              &   \ddots                              &   \vdots  \\
                    \end{bmatrix}
                \end{equation}
                which is equivalent to Eq.~(\ref{eq:C-transpose}).
        \end{subequations}
    \item To prove the following  
        \begin{subequations}
            \begin{equation}
                \left(A^{-1}\right)^{T} = \left(A^{T}\right)^{-1}
                \label{eq:ex-4-b}
            \end{equation}
            start with 
            \begin{equation}
                A^{-1} A = I
            \end{equation}
            where from  part(a) we know that 
            \begin{equation}
                \left(A^{-1}A\right)^{T} =  A^{T}\left(A^{-1}\right)^{T} = I^{T}
            \end{equation}
            and since
            \begin{equation}
                I^{T} = I 
            \end{equation}
            then
            \begin{equation}
                A^{-1}A = A^{T}\left(A^{-1}\right)^{T}= I
            \end{equation}                  
            Thus, 
            \begin{empheq}[box=\widefbox]{equation}
                \left(A^{-1}\right)^{T} =\left( A^{T}\right)^{-1}
            \end{empheq}
        \end{subequations}

    \item Lets consider the following symmetric matrix  
        \begin{subequations}
            \begin{equation}
                C   = 
                \begin{bmatrix}
                    c_{1,1}     &   c_{1,2}     &   c_{1,3}     &   \cdots  &   c_{1,n}     \\
                    c_{1,2}     &   c_{2,2}     &   c_{2,3}     &   \cdots  &   c_{2,n}     \\
                    c_{1,3}     &   c_{2,3}     &   c_{3,3}     &   \cdots  &   c_{3,n}     \\
                    \vdots      &   \vdots      &   \ddots      &   \vdots  &   \vdots      \\
                    c_{1,n}     &   c_{2,n}     &   c_{3,n}     &   \cdots  &   c_{n,n}
                \end{bmatrix}
            \end{equation}
            where for in index notation $C_{ij} = C_{ji}$. Therefore,
            \begin{empheq}[box=\widefbox]{equation}
                C^{T}   = 
                \begin{bmatrix}
                    c_{1,1}     &   c_{1,2}     &   c_{1,3}     &   \cdots  &   c_{1,n}     \\
                    c_{1,2}     &   c_{2,2}     &   c_{2,3}     &   \cdots  &   c_{2,n}     \\
                    c_{1,3}     &   c_{2,3}     &   c_{3,3}     &   \cdots  &   c_{3,n}     \\
                    \vdots      &   \vdots      &   \ddots      &   \vdots  &   \vdots      \\
                    c_{1,n}     &   c_{2,n}     &   c_{3,n}     &   \cdots  &   c_{n,n}
                \end{bmatrix}
                = C
            \end{empheq}
    \end{subequations}
\end{enumerate}
