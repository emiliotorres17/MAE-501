\section{Exercise 1}
No ``final solutions'' are provided for these types of questions, since the
whole point of them is to encourage you to express \emph{briefly} but
\emph{clearly} and \emph{in your own words} what you understand. As
explained in the directions, definitions taken from text books or the
internet do not reflect a good understanding of these terms, nor do
extremely long explanations. Equations do not express the meaning of these,
nor do literal word translations of equations show that you know what they
mean. Instead, we are looking for clear evidence that you understand what
each term means. Possible definitions for each term is provided below: 
\begin{mdframed}[style=MyFrame]
\begin{enumerate}[label=(\alph*)]
    \item Direct elimination: A method to solve a system of equations that
        uses row operations to perform forward elimination  in order to
        solve the system using backwards substitution.

    \item $LU$ Factorization: Solving method that factors the initial
        matrix into lower and upper matrices in order to simplify the
        solving the system into forward and backwards substitutions.

    \item Inverse matrix: A matrix such that any square matrix multiplied
        by its inverse gives the identity matrix.
        
    \item Symmetric matrix: Any matrix that if you switch the rows and the
        columns you do not change the matrix.

    \item Transpose: A matrix operation that  flips a matrix along its
        diagonal, i.e. switches its rows and columns.
        
    \item Permutation: Matrix operations that change the rows of an another
        matrix by multiplying it by a set of permutation matrices which are
        composed of the rows of the identity matrix. 

    \item Inner product: A matrix operation that gives the projection of
        one matrix onto another.  

    \item Singular matrix: A square matrix without an inverse.
\end{enumerate}
\end{mdframed}
