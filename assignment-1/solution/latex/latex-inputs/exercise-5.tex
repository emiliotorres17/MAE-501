\section{Exercise 5}
\begin{enumerate}[label=(\alph*)]
    \item    
        \begin{subequations}
            \begin{equation}
                A =
                \begin{bmatrix}
                    1       &       3       &       2   \\
                    3       &       5       &       4   \\
                    2       &       4       &       6   \\
                \end{bmatrix}
            \end{equation}
            As covered in lecture we know that $A$ can be factored into
            a product of lower and upper matrices, namely 
            \begin{equation}
                A = LU =
                \begin{bmatrix}
                    1       &       0       &       0   \\
                    l_{2,1} &       1       &       0   \\
                    l_{3,1} &       l_{3,2} &       1   \\
                \end{bmatrix}
                \begin{bmatrix}
                    u_{1,2} &       u_{1,2}     &       u_{1,2}     \\
                    0       &       u_{2,2}     &       u_{2,2}     \\
                    0       &       0           &       u_{3,2}     \\
                \end{bmatrix}
            \end{equation}
            where the non-zero off diagonal entries of $L$ are the
            multipliers used in forward elimination, and the entries of $U$
            are the values after all the row operations are performed.
            Therefore, the easiest way to perform $LU$ decomposition is to
            set $U=A$ and perform forward elimination. Starting with the
            first row, since $r_{2}-3r_{1}$ zeros out $a_{2,1}$, we know
            that $l_{2,1} = 3$, and that the updated $U$ matrix is 
            \begin{equation}
                U =
                \begin{bmatrix}
                    1       &       3       &       2   \\
                    0       &       -4      &      -2   \\
                    2       &       4       &       6   \\
                \end{bmatrix}
            \end{equation}
            To cancel out the third row we need $r_{3}-2r_{1}$, therefore
            $l_{31} = 2$ and the updated $U$ is
            \begin{equation}
                U =
                \begin{bmatrix}
                    1       &       3       &       2   \\
                    0       &      -4       &      -2   \\
                    0       &      -2       &       2   \\
                \end{bmatrix}
            \end{equation}
            Lastly, $r_{3}-\frac{1}{2}r_{2}$ would zero out the last entry,
            therefore $l_{32}= 1/2$, and
            \begin{equation}
                U =
                \begin{bmatrix}
                    1       &       3       &       2   \\
                    0       &      -4       &      -2   \\
                    0       &      0        &       3   
                \end{bmatrix}
            \end{equation}
            Thus
            \begin{empheq}[box=\widefbox]{equation}
               A = LU=
                \begin{bmatrix}
                    1       &      0        &       0       \\
                    3       &      1        &       0       \\
                    2       &      1/2      &       1       
                \end{bmatrix}
                \begin{bmatrix}
                    1       &      3        &       2   \\
                    0       &      -4       &      -2   \\
                    0       &      0        &       3   
                \end{bmatrix}
            \end{empheq}
            Furthermore, we can obtain the $A=LDU$ factorization, by
            dividing $U$ by a diagonal matrix $D$ that contains the pivots,
            namely
            \begin{empheq}[box=\widefbox]{equation}
               A = LDU =
                \begin{bmatrix}
                    1       &      0        &       0       \\
                    3       &      1        &       0       \\
                    2       &      1/2      &       1       \\
                \end{bmatrix}
                \begin{bmatrix}
                    1       &      0        &       0       \\
                    0       &      -4       &       0       \\
                    0       &      0        &       3       \\
                \end{bmatrix}
                \begin{bmatrix}
                    1       &      3        &       2   \\
                    0       &      1        &      -2   \\
                    0       &      0        &       1   \\
                \end{bmatrix}
            \end{empheq}
        \end{subequations}
    \item 
        \begin{subequations}
            \begin{equation}
                B =
                \begin{bmatrix}
                    1           &       0       &       1   \\
                    2           &       2       &       2   \\
                    3           &       4       &       5   
                \end{bmatrix}
            \end{equation}
            Using the same methodology as part(a), we use
            $r_{2}-2r_{1}$  which gives $l_{21}=2$ and
            \begin{equation}
                U = 
                \begin{bmatrix}
                    1           &       0       &       1   \\
                    0           &       2       &       0   \\
                    3           &       4       &       5   
                \end{bmatrix}
            \end{equation}
            Next, we need $r_{3}-3r_{1}$, therefore $l_{31}=3$ and 
            \begin{equation}
                U = 
                \begin{bmatrix}
                    1           &       0       &       1   \\
                    0           &       2       &       0   \\
                    0           &       4       &       -1   
                \end{bmatrix}
            \end{equation}
            Lastly, we need $r_{3}-2r_{2}$ giving $l_{32}=2$ and 
            \begin{equation}
                U = 
                \begin{bmatrix}
                    1           &       0       &       1   \\
                    0           &       2       &       0   \\
                    0           &       0       &       -1   
                \end{bmatrix}
            \end{equation}
            Thus,
            \begin{empheq}[box=\widefbox]{equation}
                B = LU =
                \begin{bmatrix}
                    1           &       0       &       0   \\
                    2           &       1       &       0   \\
                    3           &       2       &       1   
                \end{bmatrix}
                \begin{bmatrix}
                    1           &       0       &       1   \\
                    0           &       2       &       0   \\
                    0           &       0       &       -1   
                \end{bmatrix}
            \end{empheq}
            and
            \begin{empheq}[box=\widefbox]{equation}
                B = LDU =
                \begin{bmatrix}
                    1           &       0       &       0   \\
                    2           &       1       &       0   \\
                    3           &       2       &       1   
                \end{bmatrix}
                \begin{bmatrix}
                    1           &       0       &       0   \\
                    0           &       2       &       0   \\
                    0           &       0       &       -1   
                \end{bmatrix}
                \begin{bmatrix}
                    1           &       0       &       1   \\
                    0           &       1       &       0   \\
                    0           &       0       &       1   
                \end{bmatrix}
            \end{empheq}
        \end{subequations}
            
        \item 
            \begin{subequations}
                \begin{equation}
                    \begin{bmatrix}
                        2       &       1       &   3   \\
                        3       &       2       &   3   \\
                        3       &       1       &   0 
                    \end{bmatrix}
                \end{equation}
                First, since there is a zero pivot in the third row 
                we know were going to need a permutation matrix. A good
                chose for this problem would be to make the following row
                switches, 
                \begin{equation}
                    r_{1} \rightarrow r_{3}
                \end{equation}
                \begin{equation}
                    r_{2} \rightarrow r_{1}
                \end{equation}
                \begin{equation}
                    r_{1} \rightarrow r_{3}
                \end{equation}
                Therefore we get 
                \begin{equation}
                    PA = 
                    \begin{bmatrix}
                        0       &       1       &   0   \\
                        0       &       0       &   1   \\
                        1       &       0       &   0  
                    \end{bmatrix}
                    \begin{bmatrix}
                        2       &       1       &   3   \\
                        3       &       2       &   3   \\
                        3       &       1       &   0 
                    \end{bmatrix}
                    =
                    \begin{bmatrix}
                        3       &       2       &   3   \\
                        3       &       1       &   0   \\
                        2       &       1       &   3  
                    \end{bmatrix}
                \end{equation}
                Now we can apply the same methodology as above, starting
                with $r_{2}-r_{1}$, therefore $l_{2,1} = 1$ and  
                \begin{equation}
                    U =
                    \begin{bmatrix}
                        3       &       2       &   3       \\
                        0       &       -1      &   -3      \\
                        2       &       1       &   3  
                    \end{bmatrix}
                \end{equation}
                Next, $r_{3}-\frac{2}{3}r_{1}$, therefore $l_{3,1} = \frac{2}{3}$
                and 
                \begin{equation}
                    U = 
                    \begin{bmatrix}
                        3       &       2       &   3       \\
                        0       &       -1      &   -3      \\
                        0       &       -1/3    &   1  
                    \end{bmatrix}
                \end{equation}
                Lastly, $r_{3}-\frac{1}{3}r_{3}$, therefore $l_{3,2} =
                \frac{1}{3}$ and
                \begin{equation}
                    U = 
                    \begin{bmatrix}
                        3       &       2       &   3       \\
                        0       &       -1      &   -3      \\
                        0       &       0       &   -2  
                    \end{bmatrix}
                \end{equation}
                Thus,
                \begin{empheq}[box=\widefbox]{equation}
                    PA = LU = 
                    \begin{bmatrix}
                        1       &       0       &   0       \\
                        1       &       1       &   0       \\
                        2/3     &       1/3     &   1  
                    \end{bmatrix}
                    \begin{bmatrix}
                        3       &       2       &   3       \\
                        0       &       -1      &   -3      \\
                        0       &       0       &   -2  
                    \end{bmatrix}
                \end{empheq}
                and
                \begin{empheq}[box=\widefbox]{equation}
                    PA = LDU = 
                    \begin{bmatrix}
                        1       &       0       &   0       \\
                        1       &       1       &   0       \\
                        2/3     &       1/3     &   1  
                    \end{bmatrix}
                    \begin{bmatrix}
                        3       &       0       &   0       \\
                        0       &       -1      &   0       \\
                        0       &       0       &   -2  
                    \end{bmatrix}
                    \begin{bmatrix}
                        1       &       2       &   3       \\
                        0       &       1      &   -3      \\
                        0       &       0       &   1  
                    \end{bmatrix}
                \end{empheq}
            \end{subequations}
\end{enumerate}
