\section{Exercise 1}
\subsection{Part a}

No ``final solutions'' are provided for these types of questions, since the
whole point of them is to encourage you to express \emph{briefly} but
\emph{clearly} and \emph{in your own words} what you understand. As
explained in the directions, definitions taken from text books or the
internet do not reflect a good understanding of these terms, nor do
extremely long explanations. Equations do not express the meaning of these,
nor do literal word translations of equations show that you know what they
mean. Instead, we are looking for clear evidence that you understand what
each term means. Possible definitions for each term is provided below: 

\begin{mdframed}[style=MyFrame]
    \begin{enumerate}[label=\arabic*.]
        \item Determinants
        \item Cramer's rule
        \item Meaning and significance of $\text{det}A=0$
        \item Relationship between the determinant and volume
    \end{enumerate}
\end{mdframed}

\newcommand{\deter}[1]{\text{det}(#1)}
\subsection{Part b}
Use the determinant rules to compute the specific determinants.
\begin{enumerate}[label=\arabic*.]
    \item Reduce $\mathbf{A}$ to $\mathbf{U}$ and show $\deter{A} = \text{
            products of the pivots}$.
        \begin{equation}
            \mathbf{A} = 
            \begin{bmatrix}
                1       &       1       &       1   \\
                1       &       2       &       2   \\
                1       &       2       &       3
            \end{bmatrix}
        \end{equation}
        and
        \begin{equation}
            \mathbf{A} =
            \begin{bmatrix}
                1       &       2       &       3   \\
                2       &       2       &       3   \\
                3       &       3       &       3
            \end{bmatrix}
        \end{equation}
        \begin{mdframed}[style=MyFrame]
            We start by finding the determinant of matrix $A$ using the
            following relationship
            \begin{equation}
                \text{det}A = 
                a_{11} \text{det}
                \begin{bmatrix}
                    a_{22}      &   a_{23}  \\
                    a_{32}      &   a_{33}
                \end{bmatrix}
                -
                a_{12} \text{det}
                \begin{bmatrix}
                    a_{21}      &   a_{23}  \\
                    a_{31}      &   a_{33}
                \end{bmatrix}
                +
                a_{13} \text{det}
                \begin{bmatrix}
                    a_{21}      &   a_{22}  \\
                    a_{31}      &   a_{32}
                \end{bmatrix}
                \label{eq:a-det}
            \end{equation}
            This gives
            \begin{equation}
                \text{det}A =
                 1(6-4)-1(3-2)+1(2-2)= 1
            \end{equation}
            Next we can verify that 
            \begin{equation}
                \text{det}A =
                \prod_{i=1}^{3} u_{ii}
                \label{eq:products}
            \end{equation}
            by first factoring $A$ into a lower and upper triangular
            matrices, 
            \begin{equation}
                A = 
                \begin{bmatrix}
                    1       &   0   &   0   \\
                    1       &   1   &   0   \\
                    1       &   1   &   1   
                \end{bmatrix}
                \begin{bmatrix}
                    1   &   1   &   1   \\
                    0   &   1   &   1   \\
                    0   &   0   &   1
                \end{bmatrix}
            \end{equation}
            Therefore Eq.~(\ref{eq:products}) gives,
            \begin{equation}
                \prod_{i=1}^{3} u_{ii}
                =
                1   \cdot 1 \cdot 1 = 1
            \end{equation}
            which we can see equals $\text{det}A$. We can repeat the same
            process as above for the second part. First we use
            Eq.~(\ref{eq:a-det}) to find the determinant of $A$, namely
            \begin{equation}
                \text{det}A =
                1(6-9) - 2(6-9)  + 3(6-6) = 3
            \end{equation}
            Next we can factor $A$ into, 
            \begin{equation}
                A = 
                \begin{bmatrix}
                    0   &   1   &   0   \\
                    0   &   0   &   1   \\
                    1   &   0   &   0   
                \end{bmatrix}
                \begin{bmatrix}
                    1   &   0   &   0   \\
                    1/3 &   1   &   0   \\
                    2/3 &   0   &   1   
                \end{bmatrix}
                \begin{bmatrix}
                    3       &       3   &   3   \\
                    0       &       1   &   2   \\
                    0       &       0   &   1
                \end{bmatrix}
            \end{equation}
            Applying Eq.~(\ref{products}) we get,
            \begin{equation}
                \prod_{i=1}^{3} u_{ii}
                =
                3   \cdot 1 \cdot 1 = 3
            \end{equation}
            Thus $\text{det}A = \prod_{i}^{N}u_{ii}$.
        \end{mdframed}


    \item Apply row operations to produce an upper triangular matrix
        $\mathbf{U}$ and compute the following
        \begin{equation}
            \text{det}
            \begin{bmatrix}
                1   &   2   &   3   &   0   \\
                2   &   6   &   6   &   1   \\
                -1  &   0   &   0   &   3   \\
                0   &   2   &   0   &   7
            \end{bmatrix}
        \end{equation}
        and
        \begin{equation}
            \text{det}
            \begin{bmatrix}
                2   &   -1  &   0   &   0   \\
                -1  &   2   &   -1  &   0   \\
                0   &   -1  &   2   &   -1  \\
                0   &   0   &   -1  &   2
            \end{bmatrix}
        \end{equation}
        \begin{mdframed}[style=MyFrame]
            We start by applying the following row operations,
            \begin{enumerate}
                \item $r_{2} = r_{2} - 2r_{1}$
                \item $r_{3} = r_{3} + r_{1}$
            \end{enumerate}
            giving
            \begin{equation}
                \begin{bmatrix}
                    1   &   2   &   3   &   0   \\
                    0   &   2   &   0   &   1   \\
                    0   &   2   &   3   &   3   \\
                    0   &   2   &   0   &   7   \\
                \end{bmatrix}
            \end{equation}
            Next we zero out the second column using
            \begin{enumerate}
                \item $r_{3} = r_{3} - r_{2}$
                \item $r_{4} = r_{4} - r_{2}$
            \end{enumerate}
            giving
            \begin{equation}
                \begin{bmatrix}
                    1   &   2   &   3   &   0   \\
                    0   &   2   &   0   &   1   \\
                    0   &   0   &   3   &   2   \\
                    0   &   0   &   0   &   6   \\
                \end{bmatrix}
            \end{equation}
            Applying Eq.~(\ref{eq:products}) we get,
            \begin{equation}
                \prod_{i}^{4} = u_{ii}
                =
                1 \cdot 2 \cdot 3 \cdot 6
                =
                \text{det}
                \begin{bmatrix}
                    1   &   2   &   3   &   0   \\
                    2   &   6   &   6   &   1   \\
                    -1  &   0   &   0   &   3   \\
                    0   &   2   &   0   &   7
                \end{bmatrix}
                =
                36
            \end{equation}
            For part b we can perform the following row operations,
            \begin{enumerate}
                \item $r_{2} = r_{2} + \frac{1}{2}r_{1}$
                \item $r_{3} = r_{3} + \frac{2}{3}r_{2}$
                \item $r_{4} = r_{4} + \frac{3}{4}r_{3}$
            \end{enumerate}
            to get,
            \begin{equation}
                \begin{bmatrix}
                    2   &   -1      &   0   &   0   \\
                    0   &   3/2     &   -1  &   0   \\
                    0   &   0       &   4/3 &   -1  \\
                    0   &   0       &   0   &   5/4
                \end{bmatrix}
            \end{equation}
            Taking the product of the main diagonal we get,
            \begin{equation}
                \prod_{i}^{4} = u_{ii}
                =
                2 \cdot 3/2 \cdot 4/3 \cdot 5/4
                =
                \text{det}
                \begin{bmatrix}
                    2   &   -1  &   0   &   0   \\
                    -1  &   2   &   -1  &   0   \\
                    0   &   -1  &   2   &   -1  \\
                    0   &   0   &   -1  &   2
                \end{bmatrix}
                =
                5
            \end{equation}
        \end{mdframed}

    \item Find the determinants of a rank one matrix and a skew matrix:
        \begin{equation}
            A = 
            \begin{bmatrix}
                1   \\
                2   \\
                3
            \end{bmatrix}
            \begin{bmatrix}
                1       &   -4   &   5
            \end{bmatrix}
        \end{equation}
        and 
        \begin{equation}
            K = 
            \begin{bmatrix}
                0       &       1       &       3   \\
                -1      &       0       &       4   \\
               -3       &       -4      &       0
            \end{bmatrix}
        \end{equation}
        \begin{mdframed}[style=MyFrame]
           Starting by multiplying out $A$ we get,
            \begin{equation}
                \begin{bmatrix}
                    1       &       4       &       5   \\
                    2       &       -8      &       10  \\
                    3       &       -12     &       15
                \end{bmatrix}
            \end{equation}
            Next we can apply the determinant formula for a $3 \times 3$
            matrix giving
            \begin{equation}
                \text{det}A = 
                    1(-8\cdot15 + 10\cdot 12) +
                    4(15\cdot2 - 10\cdot3) + 
                    5(-12 \cdot 2 + 8\cdot3)
                    = 0
            \end{equation}
            For part b we can directly apply the determinant formula
            giving,
            \begin{equation}
                \text{det}A = 
                    0(0+4) - 1(0+12) + 3(4-0) =
                    -12 +12 = 0
            \end{equation}
        \end{mdframed}
\end{enumerate}
