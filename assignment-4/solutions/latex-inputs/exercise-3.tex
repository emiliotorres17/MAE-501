\section{Exercise 3}
\begin{enumerate}
    \item Solve these linear equations by Cramer's Rule.
    \begin{equation}
        2x_{1} + 5x_{2} = 1
    \end{equation}
    \begin{equation}
        x_{1} + 4x_{2} = 2
    \end{equation}
    and 
    \begin{equation}
        2x_{1} +x_{2} =1
    \end{equation}
    \begin{equation}
        x_{1} + 2x_{2} + x_{3} = 0
    \end{equation}
    \begin{equation}
        x_{2} + 2x_{3} = 0
    \end{equation}
        \begin{mdframed}[style=MyFrame]
            We can solve the two using Cramer's Rule,
            \begin{equation}
                x_{1} = \frac{\text{det}B_{1}}{\text{det}A}
                \hspace{1.0cm}
                x_{2} = \frac{\text{det}B_{2}}{\text{det}A}
                \hspace{1.0cm}
                x_{n} = \frac{\text{det}B_{n}}{\text{det}A}
            \end{equation}
            by first expressing the system of equations as matrices,
            \begin{equation}
                \begin{bmatrix}
                    2       &       5       \\
                    1       &       4
                \end{bmatrix}
                \begin{bmatrix}
                    x_{1}   \\
                    x_{2}
                \end{bmatrix}
                =
                \begin{bmatrix}
                    1   \\
                    2
                \end{bmatrix}
            \end{equation}
            Applying Cramer's rule we get the following
            \begin{equation}
                x_{1} = \frac{
                            \text{det}
                            \begin{bmatrix}
                                1   &   5   \\
                                2   &   4
                            \end{bmatrix}
                        }
                        {
                            \text{det}
                            \begin{bmatrix}
                                2   &   5   \\
                                1   &   4
                            \end{bmatrix}
                        }
                        =
                        \frac{-6}{3} = -2
            \end{equation}
            and
            \begin{equation}
                x_{2} = \frac{
                            \text{det}
                            \begin{bmatrix}
                                2   &   1   \\
                                1   &   2
                            \end{bmatrix}
                        }
                        {
                            \text{det}
                            \begin{bmatrix}
                                2   &   5   \\
                                1   &   4
                            \end{bmatrix}
                        }
                        =
                        \frac{3}{3} = 1
            \end{equation}
            Applying Cramer's rule to the next system we get,
            \begin{equation}
                x_{1} = \frac{
                            \text{det}
                            \begin{bmatrix}
                                1   &   1   &   0   \\
                                0   &   2   &   1   \\
                                0   &   1   &   2
                            \end{bmatrix}
                        }
                        {
                            \text{det}
                            \begin{bmatrix}
                                1   &   1   &   0   \\
                                1   &   2   &   1   \\
                                0   &   1   &   2
                            \end{bmatrix}
                        }
                        =
                        \frac{3}{1} = 3
            \end{equation}
            \begin{equation}
                x_{2} = \frac{
                            \text{det}
                            \begin{bmatrix}
                                1   &   1   &   0   \\
                                0   &   2   &   1   \\
                                0   &   1   &   2
                            \end{bmatrix}
                        }
                        {
                            \text{det}
                            \begin{bmatrix}
                                1   &   1   &   0   \\
                                1   &   2   &   1   \\
                                0   &   1   &   2
                            \end{bmatrix}
                        }
                        =
                        \frac{-2}{1} = -2
            \end{equation}
            and
            \begin{equation}
                x_{3} = \frac{
                            \text{det}
                            \begin{bmatrix}
                                1   &   1   &   1   \\
                                2   &   2   &   0   \\
                                0   &   1   &   0
                            \end{bmatrix}
                        }
                        {
                            \text{det}
                            \begin{bmatrix}
                                1   &   1   &   0   \\
                                1   &   2   &   1   \\
                                0   &   1   &   2
                            \end{bmatrix}
                        }
                        =
                        \frac{1}{1} = 1
            \end{equation}
        \end{mdframed}

    \item Cramer's Rule breaks down when $\text{det}A=0$. Example (a) has
        no solution while (b) has infinitely many. What are the ratios
        $x_{j} = \text{det}B/\text{det}A$ in these two cases?
        \begin{enumerate}[label=\alph*.]
            \item Parallel lines 
                \begin{equation}
                    2x_{1} + 3x_{2} = 1
                \end{equation}
                \begin{equation}
                    4x_{1} + 6x_{2} = 1
                \end{equation}
                
            \item Same line
                \begin{equation}
                    2x_{1} + 3x_{2} = 1
                \end{equation}
                \begin{equation}
                    4x_{1} + 6x_{2} =2 
                \end{equation}
        \end{enumerate}

        \begin{mdframed}[style=MyFrame]
            We can find the ratio $x_{j} = \text{det}B/\text{det}A$ using
            the following,
            \begin{equation}
                x_{1} = \frac{
                            \text{det}
                            \begin{bmatrix}
                                1   &   3   \\
                                1   &   6   
                            \end{bmatrix}
                        }
                        {
                            \text{det}
                            \begin{bmatrix}
                                2   &   3   \\
                                4   &   6   
                            \end{bmatrix}
                        }
                        =
                        \frac{3}{0}
            \end{equation}
            and
            \begin{equation}
                x_{2} = \frac{
                            \text{det}
                            \begin{bmatrix}
                                2   &   1   \\
                                4   &   1   
                            \end{bmatrix}
                        }
                        {
                            \text{det}
                            \begin{bmatrix}
                                2   &   3   \\
                                4   &   6   
                            \end{bmatrix}
                        }
                        =
                        \frac{-2}{0}
            \end{equation}
        For the same line case we get the following,
            \begin{equation}
                x_{1} = \frac{
                            \text{det}
                            \begin{bmatrix}
                                1   &   3   \\
                                2   &   6   
                            \end{bmatrix}
                        }
                        {
                            \text{det}
                            \begin{bmatrix}
                                2   &   3   \\
                                4   &   6   
                            \end{bmatrix}
                        }
                        =
                        \frac{0}{0}
            \end{equation}
            and
            \begin{equation}
                x_{2} = \frac{
                            \text{det}
                            \begin{bmatrix}
                                2   &   1   \\
                                4   &   2   
                            \end{bmatrix}
                        }
                        {
                            \text{det}
                            \begin{bmatrix}
                                2   &   3   \\
                                4   &   6   
                            \end{bmatrix}
                        }
                        =
                        \frac{0}{0}
            \end{equation}
        \end{mdframed}


    \item Find $A^{-1}$ from the cofactor formula $C^{T}/\text{det}A$.
        \begin{enumerate}[label=\alph*.]
            \item 
                \begin{equation}
                    A =
                    \begin{bmatrix}
                        1 & 2 & 0 \\
                        0 & 3 & 0 \\
                        0 & 7 & 1
                    \end{bmatrix}
                \end{equation}
                \begin{mdframed}[style=MyFrame]
                    Starting by finding the cofactor matrix,
                    \begin{equation}
                        C = 
                        \begin{bmatrix}
                            3   &   0       &   0   \\
                            -2  &   1       &   -7  \\
                            0   &   0       &   3   \\
                        \end{bmatrix}
                        \end{equation}
                        Next using $C$ we can find $\text{det}A$,
                        \begin{equation}
                            \text{det}A = 1\cdot 3 + 2 \cdot 0 + 0\cdot 0
                                            = 3
                        \end{equation}
                        Thus,
                        \begin{equation}
                            A^{-1} = 
                            \frac{1}{3}
                            \begin{bmatrix}
                                3       &   -2  &   0   \\
                                0       &   1   &   0   \\
                                0       &   -7  &   3
                            \end{bmatrix}
                        \end{equation}
                    \end{mdframed}
            \item 
                \begin{equation}
                    A =
                    \begin{bmatrix}
                        2 & -1 & 0 \\
                        -1 & 2 & -1 \\
                        0 & -1 & 2
                    \end{bmatrix}
                \end{equation}
                \begin{mdframed}[style=MyFrame]
                    Repeating the process from above we get,
                    \begin{equation}
                        C = 
                        \begin{bmatrix}
                            3       &       2       &       1   \\
                            2       &       4       &       2   \\
                            1       &       2       &       3   
                        \end{bmatrix}
                    \end{equation}
                    and 
                    \begin{equation}
                        \text{det}A = 2 \cot 3 + -1\cdot2 + 0 \cdot 1 
                                        = 4
                    \end{equation}
                    Thus,
                    \begin{equation}
                        A^{-1} = 
                        \frac{1}{4}
                        \begin{bmatrix}
                            3       &       2       &       1   \\
                            2       &       4       &       2   \\
                            1       &       2       &       3   
                        \end{bmatrix}
                    \end{equation}
                \end{mdframed}


        \end{enumerate}
        
    \item If all entries of $A$ and $A^{-1}$ are integers, prove that
        $\text{det}A=1$ or $-1$.

    \item Find the following:
        \begin{enumerate}[label=\alph*.]
            \item Find the area of the parallelogram with edges $v=(3,2)$
                and $w=(1,4)$
            \item Find the area of the triangle with sides $v$, $w$, and
                $v+w$. 
            \item Find the area of the triangle with sides $v$, $w$, and $w-v$.
        \end{enumerate}
\end{enumerate}
