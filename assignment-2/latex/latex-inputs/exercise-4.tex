\section{Exercise 4}
As discussed in lecture tri-diagonal solvers are highly efficient
solvers used to solve a particular system of equations, including
diffusion transport type problems, which we will explore in the following
exercises. 
\begin{enumerate}[label=(\alph*)]
    \item Start with a brief description of tri-diagonal solvers,
        highlighting their advantages and disadvantages.
        
    \item Using the following three steps
        \begin{enumerate}[label=(\arabic*)]
            \item Store four 1D vectors 
            \item Elimination
                \begin{algorithmic}
                    \FOR{i = 1 to N}
                    \STATE b(i) = b(i) - c(i-1)*a(i)/b(i-1)
                    \STATE d(i) = d(i) - d(i-1)*a(i)/b(i-1)
                    \ENDFOR
                \end{algorithmic}
            \item Back substitution 
                \begin{algorithmic}
                    \STATE d(N) = d(N)/b(N)
                    \FOR{i = N-1 to 0}
                    \STATE d(i) = (d(i) - c(i)*d(i+1))/b(i))
                    \ENDFOR
                \end{algorithmic}
        \end{enumerate}
    \emph{\textbf{hand write}} pseudo code for a Python tri-diagonal solver
        subroutine. Make sure to include a brief description of the
        boundary treatment in your pseudo code. Upload your
        \emph{\textbf{hand written subroutine}} as part of your submission
        for this problem.
    
    \item Use your pseudo code from part(b) to code a tri-diagonal solver
        that takes in four 1D-vectors and outputs a single 1D solutions
        vector.  Upload a well annotated code with a single verification
        cases as your submission to this problem.
\end{enumerate}
