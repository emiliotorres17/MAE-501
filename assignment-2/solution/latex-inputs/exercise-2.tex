\section{Exercise 2}
For each of the three matrices perform the following:
\begin{enumerate}[label=(\alph*)]
    \item $LU$ factorization using augmented matrix form.
    \item Find the reduced row echelon form $R$.
    \item Identify the pivots and free variables, as well as find the rank.
    \item Give basis of vectors for the column space, nullspace, row space,
        and left null space, and their respective dimensions.
\end{enumerate}
\subsection{Part a.}
\begin{equation}
    A =
    \begin{bmatrix}
        1       &   -1  &   0    \\
        -1      &   2   &   -1   \\
        1       &   -1  &   2
    \end{bmatrix}
\end{equation}
Starting with the augmented $A^{\ast}$ matrix
\begin{equation}
    A^{\ast} =
    \begin{bmatrix}[ccc|ccc]
        1       &   -1  &   0   &   1   &   0   &   0   \\
        -1      &   2   &   -1  &   0   &   1   &   0   \\
        1       &   -1  &   2   &   0   &   0   &   1
    \end{bmatrix}
\end{equation}
and performing both $r_{2}\rightarrow r_{1} + r_{2}$ and $r_{3}\rightarrow
r_{3}-r_{1}$ to find the $LU$ factors,   
\begin{equation}
    A^{\ast} =
    \begin{bmatrix}[ccc|ccc]
        1       &   -1  &   0   &   1   &   0   &   0   \\
        0       &   1   &   -1  &   1   &   1   &   0   \\
        0       &   0   &   2   &   -1   &   0   &   1
    \end{bmatrix}
\end{equation}
Thus,
\begin{mdframed}[style=MyFrame]
    \begin{equation}
        L =
        \begin{bmatrix}
            1       &   0   &   0    \\
            -1      &   1   &   0   \\
            1       &   0  &   1
        \end{bmatrix}
    \end{equation}
    and
    \begin{equation}
        U =
        \begin{bmatrix}
            1       &   -1  &   0       \\
            0       &   1   &   -1      \\
            0       &   0   &   2
        \end{bmatrix}
    \end{equation}
\end{mdframed}
Furthermore, we can complete the elimination using $r_{2}\rightarrow
\frac{1}{2}r_{3} + r_{2}$ and $r_{1} \rightarrow r_{1} + r_{2}$, giving 
\begin{equation}
    A^{\ast} =
    \begin{bmatrix}[ccc|ccc]
        1       &   0  &   0    &   3/2     &   1   &   1/2   \\
        0       &   1   &  0    &   1/2     &   1   &   1/2   \\
        0       &   0   &   2   &   -1      &   0   &   1
    \end{bmatrix}
\end{equation}
Thus,
\begin{mdframed}[style=MyFrame]
    \begin{equation}
        \text{rref}(A) =
        \begin{bmatrix}
            1       &      0    &   0   \\
            0       &      1    &   0   \\
            0       &      0    &   1
        \end{bmatrix}
    \end{equation}
\end{mdframed}
\begin{mdframed}[style=MyFrame]
    Furthermore, from the $\text{rref}(A)$ we see that there are no free
    variables, and the matrix is full rank. Thus, $N(A) = \vec{0}$ and
    \begin{equation}
        C(A) =
        \left\{
            \begin{bmatrix}
                1   \\
                -1  \\
                1   
            \end{bmatrix}, 
            \begin{bmatrix}
                -1   \\
                2    \\
                -1   
            \end{bmatrix},
            \begin{bmatrix}
                0   \\
                -1  \\
                2   
            \end{bmatrix}
           \right\}
    \end{equation}
    and 
    \begin{equation}
        C\left(A^{T}\right) =
        \left\{
            \begin{bmatrix}
                1   \\
                -1  \\
                0   
            \end{bmatrix},
            \begin{bmatrix}
                -1   \\
                2    \\
                -1   
            \end{bmatrix},
            \begin{bmatrix}
                1   \\
                -1  \\
                2   
            \end{bmatrix}
           \right\}
    \end{equation}
\end{mdframed}
%-------------------------------------------------------------------------%
% Part b.                                                                 %
%-------------------------------------------------------------------------%
\subsection{Part b.}
\begin{equation}
    B =
    \begin{bmatrix}
        1           &           2       &       5   \\
        3           &           6       &       15  \\
        0           &           1       &       0   \\
        5           &           16      &       25  
    \end{bmatrix}
\end{equation}
Starting with the augmented matrix
\begin{equation}
    B^{\ast} =
    \begin{bmatrix}[ccc|cccc]
        1           &           2       &       5   &   1   &   0   &   0   &   0   \\
        3           &           6       &       15  &   0   &   1   &   0   &   0   \\   
        0           &           1       &       0   &   0   &   0   &   1   &   0   \\
        5           &           16      &       25  &   0   &   0   &   0   &   1
    \end{bmatrix}
\end{equation}
Furthermore, performing the following row operations
\begin{enumerate}[label=(\arabic*)]
    \item $r_{2} \rightarrow r_{2} - 3r_{1} $
    \item $r_{4} \rightarrow r_{4} - 5r_{1} $
    \item Switching $r_{2}$ and $r_{3}$
    \item $r_{4} \rightarrow r_{4} - 6r_{1}$
\end{enumerate}
gives,
\begin{equation}
    B^{\ast} =
    \begin{bmatrix}[ccc|cccc]
        1           &           2       &       5   &   1   &   0   &   0   &   0   \\
        0           &           1       &       0   &   0   &   0   &   1   &   0   \\   
        0           &           0       &       0   &   -3  &   1   &   0   &   0   \\
        0           &           0       &       0   &   -5  &   -6  &   0   &   1
    \end{bmatrix}
\end{equation}
which transforms our original $B$ matrix to an upper diagonal matrix.
moreover, to find the $L$ factor we have to use the matrix operations
performed to transform $B$ into an upper diagonal matrix, namely
\begin{equation}
    L^{-1}PB = U
\end{equation}
where
\begin{equation}
    L^{-1}P = E_{42} P_{23} E_{41} E_{21}
\end{equation}
Therefore we can solve for $L$ by taking the inverse of $P$ using $P^{-1} =
P^{T}$ and then taking the inverse, giving
\begin{equation}
    L^{-1} \underbrace{P P^{T}}_{I} = E_{42} P_{23} E_{41} E_{21} P_{23}^{T} 
\end{equation}
\begin{equation}
    L = P^{-1}_{23} E^{-1}_{21} E^{-1}_{41} P^{-1}_{23} E^{-1}_{42}
\end{equation}
Thus,
\begin{mdframed}[style=MyFrame]
    \begin{equation}
        PB = LU
    \end{equation}
\end{mdframed}
where
\begin{mdframed}[style=MyFrame]
    \begin{equation}
        L = 
        \begin{bmatrix}
            1   &   0   &   0   &   0   \\
            0   &   1   &   0   &   0   \\
            3   &   0   &   1   &   0   \\
            5   &   6   &   0   &   1
        \end{bmatrix}
    \end{equation}
\end{mdframed}
and 
 \begin{mdframed}[style=MyFrame]
    \begin{equation}
        U = 
        \begin{bmatrix}
            1   &   2   &   5       \\
            0   &   1   &   0       \\
            0   &   0   &   0       \\
            0   &   0   &   0
        \end{bmatrix}
    \end{equation}
 \end{mdframed}
Furthermore, we can complete the elimination process using $r_{1} \rightarrow r_{1} - 2r_{2}$ giving 
\begin{equation}
    B^{\ast} =
    \begin{bmatrix}[ccc|cccc]
        1           &           0       &       5       &       1   &   0   &   0   &   0   \\       
        0           &           1       &       0       &       0   &   1   &   0   &   0   \\        
        0           &           0       &       0       &       -3  &   0   &   1   &   0   \\        
        0           &           0       &       0       &       -5  &   -6  &   0   &   1        
    \end{bmatrix}
\end{equation}
This gives the row reduced echelon form as
\begin{mdframed}[style=MyFrame]
    \begin{equation}
        \text{rref}(B) = 
        \begin{bmatrix}
            {\color{red}1}              &           0           &       {\color{green}5}     \\
             0                          &     {\color{red}1}    &       0      \\
             0                          &           0           &       0      \\
             0                          &           0           &       0      
        \end{bmatrix}
    \end{equation}
Therefore $\text{rank}(B)=2$ and the pivots and free are labelled in red and green, respectively.
\end{mdframed}




Moreover, we can express basis of the column and row space as 
\begin{mdframed}[style=MyFrame]
    \begin{equation}
        C(B) =
        \left\{
        \begin{bmatrix}
            1       \\
            3       \\
            0       \\
            5
        \end{bmatrix},
        \begin{bmatrix}
            2       \\
            6       \\
            1       \\
            16
        \end{bmatrix}
        \right\}
    \end{equation}
    and
    \begin{equation}
        C(B^{T}) =
        \left\{
        \begin{bmatrix}
            1       \\
            2       \\
            5
        \end{bmatrix},
        \begin{bmatrix}
            0       \\
            1       \\
            0
        \end{bmatrix}
        \right\}
    \end{equation}
\end{mdframed}
Lastly, we can express the basis of the nullspace and the left nullspace as
\begin{mdframed}[style=MyFrame]
    \begin{equation}
        N(B) = 
        \left\{
        \begin{bmatrix}
            -5      \\
            0       \\
            1
        \end{bmatrix}
        \right\}
    \end{equation}
    and 
    \begin{equation}
        N(B^{T}) = 
        \left\{
        \begin{bmatrix}
            -5      \\
            0       \\
            -6      \\
            1
        \end{bmatrix},
        \begin{bmatrix}
            -3      \\
            1       \\
            0       \\
            0
        \end{bmatrix}
        \right\}
    \end{equation}
\end{mdframed}
\subsection{Part c.}
\begin{equation}
    C = 
    \begin{bmatrix}
        1   &   1   &   1   \\
        1   &   -1  &   0   \\
        2   &   0   &   4
    \end{bmatrix}
\end{equation}
Starting with the augmented matrix $C^{\ast}$,
\begin{equation}
    C^{\ast} =
    \begin{bmatrix}[ccc|ccc]
        1   &   1   &   1   &   1   &  0   &   0   \\
        1   &   -1  &   0   &   0   &  1   &   0   \\
        2   &   0   &   4   &   0   &  0   &   1
    \end{bmatrix}
\end{equation}
and eliminating the first entries along the first column by performing
$r_{2} \rightarrow r_{2} - r_{1}$ and $r_{3}\rightarrow r_{3} -2 r_{1}$
giving
\begin{equation}
    C^{\ast} =
    \begin{bmatrix}[ccc|ccc]
        1   &   1   &   1   &   1   &  0   &   0   \\
        0   &   -2  &   -1  &   -1  &  1   &   0   \\
        0   &   -2   &   2  &   -2  &  0   &   1
    \end{bmatrix}
\end{equation}
Next we eliminate the entries under the second pivot $r_{3} \rightarrow r_{3}
- r_{2}$, giving
\begin{equation}
    C^{\ast} =
    \begin{bmatrix}[ccc|ccc]
        1   &   1   &   1   &   1   &  0   &   0   \\
        0   &   -2  &   -1  &   -1  &  1   &   0   \\
        0   &   0   &   3   &   -1  &  -1  &   1
    \end{bmatrix}
\end{equation}
Thus, 
\begin{mdframed}[style=MyFrame]
    \begin{equation}
        L =
        \begin{bmatrix}
            1       &   0   &   0   \\
            1       &   1   &   0   \\
            2       &   1   &   1
        \end{bmatrix}
    \end{equation}
    \begin{equation}
        U =
        \begin{bmatrix}
            1       &   1   &   1   \\
            0       &   -2  &   -1  \\
            0       &   0   &   3
        \end{bmatrix}
    \end{equation}
\end{mdframed}
Next we can complete the elimination using the following row operations:
\begin{enumerate}
    \item $r_{2} \rightarrow r_{2} + \frac{1}{3}r_{3}$
    \item $r_{1} \rightarrow r_{1} - \frac{1}{3}r_{3}$
    \item $r_{1} \rightarrow r_{1} - \frac{1}{2}r_{2}$
\end{enumerate}
This give
\begin{equation}
    C^{\ast} =
    \begin{bmatrix}[ccc|ccc]
        1   &   0   &   0   &   2/3     &  2/3  &   -1/6    \\
        0   &   -2  &   0   &   -4/3    &  2/3  &   1/3     \\
        0   &   0   &   3   &   -1      &  -1   &   1
    \end{bmatrix}
\end{equation}
Thus,
\begin{mdframed}[style=MyFrame]
    \begin{equation}
        \text{rref}(C) =
        \begin{bmatrix}
            1       &       0       &   0   \\
            0       &       1       &   0   \\
            0       &       0       &   1 
        \end{bmatrix}
    \end{equation}
    Clearly from the reduced row echelon we can see that $C$ is of full
    rank and there are no free variables.
\end{mdframed}
\begin{mdframed}[style=MyFrame]
    Furthermore, since $\text{rank}(C)=3$ we can conclude that
    $N(C)=\vec{0}$ and 
        \begin{equation}
            C(C) =
            \left\{
            \begin{bmatrix}
                1   \\
                1   \\
                2
            \end{bmatrix},
            \begin{bmatrix}
                1   \\
                -1  \\
                0
            \end{bmatrix},
            \begin{bmatrix}
                1   \\
                0  \\
                4
            \end{bmatrix}
            \right\}
        \end{equation}
        and 
        \begin{equation}
            C(C^{T}) =
            \left\{
            \begin{bmatrix}
                1   \\
                1   \\
                1
            \end{bmatrix},
            \begin{bmatrix}
                1   \\
                -1  \\
                0
            \end{bmatrix},
            \begin{bmatrix}
                2   \\
                0  \\
                4
            \end{bmatrix}
            \right\}
    \end{equation}
\end{mdframed}

