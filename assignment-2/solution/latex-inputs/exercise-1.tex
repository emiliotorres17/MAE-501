\section{Exercise 1}
\subsection{Part a.}
Which of the following subsets of $\mathbb{R}^{3}$ are actually subspaces?
If not, explain why.
\begin{enumerate}[label=(\arabic*)]
    %---------------------------------------------------------------------%
    % 1.                                                                  %
    %---------------------------------------------------------------------%
    \item The plane of vectors $\mathbf{v} = (v_{1}, v_{2}, v_{3})$ with
        first component $v_{1}=0$.
        \newline
        \begin{mdframed}[style=MyFrame]
        Start with two vectors in the subset 
        \begin{equation}
            \mathbf{v}_{1} =  (0,1,2)
        \end{equation}
        \begin{equation}
            \mathbf{v}_{2} =  (0,7,5)
        \end{equation}
        and verify that the addition of the vectors are in the space,
        namely
        \begin{equation}
            \mathbf{v}_{1} + \mathbf{v}_{2} = (0,8,7)
            \label{eq:ex-a-1-1}
        \end{equation}
        Next, we can verify that $c\mathbf{v}$ is also in the subspace of
        the vectors,
        \begin{equation}
            c\mathbf{v}_{1} =  (0,c,2c)
            \label{eq:ex-a-1-2}
        \end{equation}
            Since the results shown in Eqs.~(\ref{eq:ex-a-1-1}) \&
            (\ref{eq:ex-a-1-2}) are clearly in the subspace then the plane
            of vectors is a subspace.
        \end{mdframed}
    %---------------------------------------------------------------------%
    % 2.                                                                  %
    %---------------------------------------------------------------------%
    \item The plane of vectors $\mathbf{v}$ with second component $v_{2}=2$.
        \newline
        \begin{mdframed}[style=MyFrame]
            Using the same procedure as above, we start by defining two
            vectors in the subspace,
            \begin{equation}
                \mathbf{v}_{1} = (0,2,0)
            \end{equation}
            \begin{equation}
                \mathbf{v}_{2} = (1,2,0)
            \end{equation}
            and checking that the addition of $\mathbf{v}_{1}$ and
            $\mathbf{v}_{2}$ are also in the subspace, namely
            \begin{equation}
                \mathbf{v}_{1} + \mathbf{v}_{2} = (1,4,0)
                \label{eq:ex-a-2-1}
            \end{equation}
            Clearly, $\mathbf{v}_{1} + \mathbf{v}_{2}$ is not in the
            plane and therefore the plane is not a subspace.
        \end{mdframed}
    %---------------------------------------------------------------------%
    % 3.                                                                  %
    %---------------------------------------------------------------------%
    \item The vectors $\mathbf{v}$ with $v_{1} = v_{2} = 0$.
        \newline
        \begin{mdframed}[style=MyFrame]
            Again we start by defining two vectors and verify that they
            are closed by addition and multiplication, namely
            \begin{equation}
                \mathbf{v}_{1} = (0,0,1)
            \end{equation}
            \begin{equation}
                \mathbf{v}_{2} = (0,0,2)
            \end{equation}
            This gives,
            \begin{equation}
                \mathbf{v}_{1} + \mathbf{v}_{2} = (0,0,3)
            \end{equation}
            \begin{equation}
                c\mathbf{v}_{1} = (0,0,3c)
            \end{equation}
            which are clearly in the subspace. Thus the vectors are a
            subspace.
        \end{mdframed}
    \item 
        The solitary vector $\mathbf{v}=(0,0,0)$
        \newline
        \begin{mdframed}[style=MyFrame]
            Clearly since $0+0=0$ and $0\cdot0=0$ the vector is a subspace.
            Furthermore, this subspace is more commonly referred to as the
            trivial subspace.
        \end{mdframed}
    \item 
        All combinations of the vector $\mathbf{x} =(1,0,0)$ and
        $\mathbf{y}=(2,0,1)$.
        \begin{mdframed}[style=MyFrame]
            We start by defining a vector with all the combinations of
            $\mathbf{x}$ and $\mathbf{y}$,
            \begin{equation}
                \mathbf{v} = \alpha \mathbf{x} + \beta \mathbf{y}
            \end{equation}
            and use this to define two more vectors in the subspace, namely
            \begin{equation}
                \mathbf{v}_{1} = \alpha_{1} \mathbf{x} + \beta_{1} \mathbf{y}
            \end{equation}
            \begin{equation}
                \mathbf{v}_{2} = \alpha_{2} \mathbf{x} + \beta_{2} \mathbf{y}
            \end{equation}
            Next we show that the subspace is closed by addition, namely
            \begin{equation}
                \mathbf{v}_{1} + \mathbf{v}_{2} = 
                        \underbrace{(\alpha_{1} + \alpha_{2})}_{\equiv \alpha_{3}} \mathbf{x} +
                        \underbrace{(\beta_{1} + \beta_{2})}_{\equiv \beta_{3}} \mathbf{y}
            \end{equation}
            Therefore since  
            $\mathbf{v}_{1} + \mathbf{v}_{2} = \alpha_{3}\mathbf{x} + \beta_{3} \mathbf{y}$ 
            produces another combination of vectors $\mathbf{x}$ and
            $\mathbf{y}$ the subspace is closed by addition.
            Lastly we can show that scalar multiplication is closed by
            multiplying $c\mathbf{v}_{1}$, giving
            \begin{equation}
                c\mathbf{v}_{1} = c\alpha_{1} \mathbf{x} + c\beta_{1} \mathbf{y}
            \end{equation}
            which again is just another combination of $\mathbf{x}$ and
            $\mathbf{y}$. Thus the vectors are a subspace.
        \end{mdframed}
    \item The vectors $\mathbf{v} =(v_{1}, v_{2}, v_{3})$ that satisfy 
        $v_{3}-v_{2}+3v_{1}=0$.
        \begin{mdframed}[style=MyFrame]
            Again we use the above procedure and start by defining two
            vectors
            \begin{equation}
                \mathbf{v}_{1} = (1,1,-2)
            \end{equation}
            \begin{equation}
                \mathbf{v}_{2} = (2,0,-6)
            \end{equation}
            and verify that they are closed by addition and scalar
            multiplication. This gives,
            \begin{equation}
                \mathbf{v}_{1} + \mathbf{v}_{2} = (3,1,-8)
            \end{equation}
            \begin{equation}
                3\mathbf{v}_{1} = (3,3,-6)
            \end{equation}
            which are both clearly in the subspace. Thus the vectors do form
            a subspace.
        \end{mdframed}
\end{enumerate}

\subsection{Part b.}

No ``final solutions'' are provided for these types of questions, since the
whole point of them is to encourage you to express \emph{briefly} but
\emph{clearly} and \emph{in your own words} what you understand. As
explained in the directions, definitions taken from text books or the
internet do not reflect a good understanding of these terms, nor do
extremely long explanations. Equations do not express the meaning of these,
nor do literal word translations of equations show that you know what they
mean. Instead, we are looking for clear evidence that you understand what
each term means. Possible definitions for each term is provided below: 
\begin{mdframed}[style=MyFrame]
    \begin{enumerate}[label=(\alph*)]
        \item Subspace: A subspace of $\mathbb{R}^{n}$ is a subset of
            vectors in $\mathbb{R}^{n}$ that is closed under addition and
            scalar multiplication.
            
        \item Column space: The column space of a matrix is the subspace of
            $\mathbb{R}^{m}$ spanned by the columns of the matrix.

        \item Row space: The row space of a matrix is the subspace of
            $\mathbb{R}^{n}$ spanned by the rows of the matrix.
            
        \item Null space: The null space is the subspace of
            $\mathbb{R}^{n}$ consisting of all the solutions of the
            homogeneous equation, and is perpendicular to the row space.

        \item Left null space: The left null space is a subspace in
            $\mathbb{R}^{m}$ which is the null space of the transpose of a
            matrix, and is perpendicular to the column space. 
    \end{enumerate}
\end{mdframed}
