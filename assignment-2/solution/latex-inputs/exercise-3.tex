\section{Exercise 3}
\subsection{Part a.}
How can you verify if a large matrix has an inverse without actually trying
to compute it?
\begin{mdframed}[style=MyFrame]
    The easiest way to verify that a large matrix has an inverse without
    actually computing is to verify that is of full rank.
\end{mdframed}
\subsection{Part b.}
For which numbers of $c$ and $d$ do the following matrix have rank 2?
\begin{equation}
    A =
    \begin{bmatrix}
        1   &   2   &   5   &   0   &   5   \\
        0   &   0   &   c   &   2   &   2   \\
        0   &   0   &   0   &   d   &   2
    \end{bmatrix}
\end{equation}

\begin{mdframed}[style=MyFrame]
    By examination we can see that the simple solution is let $c=0$ and
    $d=2$, which gives the following row reduced echelon form
    \begin{equation}
        \text{rref}(A) =
        \begin{bmatrix}
            1   &   2   &   5   &   0   &   5   \\
            0   &   0   &   0   &   1   &   1   \\
            0   &   0   &   0   &   0   &   0
        \end{bmatrix}
    \end{equation}
    
    Where we can clear see that the matrix has a rank of $2$.
\end{mdframed}

\subsection{Part c.}
What can be said about the four fundamental subspaces when $Ax=b$ has an
unique solution, no solution, and infinitely many solutions?
\begin{mdframed}[style=MyFrame]
    \begin{enumerate}[label=(\alph*)]
        \item If a system has an unique solution if, 
            \begin{enumerate}[label=(\arabic*)]
                \item $\text{dim}(N(A)) = 0$
                \item $\text{rank} = \text{dim}(C(A))=n-\text{dim}(N(A)) =
                    n$
                \item $\text{dim}(C(A^{T})) = \text{dim}(C(A)) = n $
                \item $\text{dim}(N(A^{T})) + \text{dim}(C(A^T)) = m$
                \item $\text{dim}(N(A^{T}))=m-n$
            \end{enumerate}
            
        \item If the system $Ax=b$ has no solution then the rows vectors do
            not intersect. Therefore, the dimensions of the column and row
            space would be the number of pivots $r$, where $r<n$.
            Additionally,  the null space would have dimensions $n-r>1$,
            and lastly the left null space would have dimensions of
            $m-r>m-n$. 
            
        \item Similarly, if a system has infinite solutions then the
            dimensions of the nullspace is larger than 1,
            $\text{rank}(C(A))=\text{dim}(C(A))=\text{dim}(C(A^{T}))<n$,
            etc.
    \end{enumerate}
\end{mdframed}

\subsection{Part d.}
What are the dimensions of the four fundamental subspaces for $A$, $B$, and
$C$, if $I$ is the $3\times3$ identity matrix and $0$ is the $3\times2$
zero matrix?
\begin{enumerate}[label=(\alph*)]
    \item 
        \begin{equation}
            A = 
            \begin{bmatrix}
                1   &   0   &   0   &   0   &   0   \\
                0   &   1   &   0   &   0   &   0   \\
                0   &   0   &   1   &   0   &   0
            \end{bmatrix}
        \end{equation}
        \begin{mdframed}[style=MyFrame]
            From examination
                \begin{equation}
                    \text{dim}(C(A)) = r =3
                \end{equation}
                \begin{equation}
                    \text{dim}(C(A^{T})) = r = 3
                \end{equation}
                \begin{equation}
                    \text{dim}(N(A)) = n-r = 2 
                \end{equation}
                \begin{equation}
                    \text{dim}(N(A^{T})) = m-r = 0 
                \end{equation}
        \end{mdframed}
    \item 
        \begin{equation}
            B = 
            \begin{bmatrix}
                1   &   0   &   0   & 1 &   0   &   0  \\
                0   &   1   &   0   & 0 &   1   &   0  \\
                0   &   0   &   1   & 0 &   0   &   1  \\
                0   &   0   &   0   & 0 &   0   &   0  \\
                0   &   0   &   0   & 0 &   0   &   0
                
            \end{bmatrix}
        \end{equation}
            From examination
            \begin{mdframed}[style=MyFrame]
                \begin{equation}
                    \text{dim}(C(B)) = r = 3
                \end{equation}
                \begin{equation}
                    \text{dim}(C(B^{T})) = r = 3
                \end{equation}
                \begin{equation}
                    \text{dim}(N(B)) = n-r = 3 
                \end{equation}
                \begin{equation}
                    \text{dim}(N(B^{T})) = m-r = 2 
                \end{equation}
            \end{mdframed}
    \item 
        \begin{equation}
            C = 
            \begin{bmatrix}
                0   &   0   \\
                0   &   0   \\
                0   &   0
                
            \end{bmatrix}
        \end{equation}
        \begin{mdframed}[style=MyFrame]
            From examination
                \begin{equation}
                    \text{dim}(C(C)) = r = 0
                \end{equation}
                \begin{equation}
                    \text{dim}(C(C^{T})) = r = 0
                \end{equation}
                \begin{equation}
                    \text{dim}(N(C)) = n-r = 2 
                \end{equation}
                \begin{equation}
                    \text{dim}(N(C^{T})) = m-r = 3 
                \end{equation}
        \end{mdframed}
\end{enumerate}
