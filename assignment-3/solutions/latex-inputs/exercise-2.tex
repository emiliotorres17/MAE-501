\newcommand{\matA}{\mathbf{A}}
\newcommand{\matB}{\mathbf{B}}
\section{Exercise 2}
\subsection{Part a.}
Suppose $\mathbf{A}$ is $3 \times 4$ and $\mathbf{B}$ is $4 \times 5$ and
$\matA \matB =0$. So $N(\matA)$ contains $C(\matB)$. Prove from the
dimensions of $N(\matA)$ and $C(\matB)$ that $\text{rank}(\matA) +
\text{rank}(\matB) \leq 4$. 

\begin{mdframed}[style=MyFrame]
    Since the $N(A)$ contains the $C(B)$ means the dimensions of $C(B) \leq
    \text{ dimension } N(A)$. Thus $\text{rank}(B) \leq 4 - \text{
        rank}(A)$. Furthermore, since 
    \begin{equation}
        \text{dim}(N(A)) + \text{dim}(C(A^{T})) = 
                \text{ number of columns } = 4
    \end{equation}
    and 
    \begin{equation}
        \text{dim}(C(A^{T})) = \text{rank}(A)
    \end{equation}
    we conclude
    \begin{equation}
        \text{rank}(B) \leq 4-\text{rank}(A)
    \end{equation}
\end{mdframed}

\subsection{Part b.}
For four non-zero vectors $\mathbf{r}$, $\mathbf{n}$, $\mathbf{c}$,
$\mathbf{l}$, in $\mathbb{R}^{2}$
\begin{enumerate}[label=(\alph*)]
    \item What are the conditions for those to be bases for the four
        fundamental subspaces for a $2 \times 2$ matrix?
    \item What is one possible matrix $\matA$?
\end{enumerate}
\begin{mdframed}[style=MyFrame]
    \begin{enumerate}[label=(\alph*)]
        \item We suppose that $\mathbf{r}$ is a basis vector of the row space, $\mathbf{n}$
            is a basis vector of the nullspace. The condition 
            $\text{dim}(C(A^{T})) +\text{dim}(N(A)) =2 $ 
            combined with the statement above
            requires that 
            $\text{dim}(C(A^{T}))=\text{dim}(N(A))=1$. 
            Therefore, the bases of these two subspaces, contain exactly
            one vector, $\mathbf{r}$ and $\mathbf{n}$ respectively. Since the nullspace and
            row space are orthogonal complements, we have the condition
            that all basis vectors of the row space must be perpendicular
            to all basis vectors of the nullspace. Therefore $\mathbf{r}^{T}\mathbf{n}=0$.
            Using the same argument we find the condition
            $\mathbf{c}^{T}\mathbf{l}=0$.

        \item Because this a rank 1 matrix, all possible examples have the
            form $a\mathbf{c}\mathbf{r}^{T}$ with $a \ne 0$.
    \end{enumerate}
\end{mdframed}


\subsection{Part c.}
Project the vector $\mathbf{b}$ onto the line through $\mathbf{a}$. Verify
that the residual $\mathbf{e}$ is perpendicular to $\mathbf{a}$.
\begin{equation}
    \mathbf{b} = 
    \begin{bmatrix}
        1       \\
        2       \\
        2
    \end{bmatrix}
    \hspace{2.0cm} 
    \mathbf{a} =
    \begin{bmatrix}
        1       \\
        1       \\
        1
    \end{bmatrix}
\end{equation}
and 
\begin{equation}
    \mathbf{b} = 
    \begin{bmatrix}
        1       \\
        3       \\
        1
    \end{bmatrix}
    \hspace{2.0cm} 
    \mathbf{a} =
    \begin{bmatrix}
        -1       \\
        -3       \\
        -1
    \end{bmatrix}
\end{equation}

\begin{mdframed}[style=MyFrame]
    We start by finding the projection of $\mathbf{b}$ onto $\mathbf{a}$,
    \begin{equation}
        P_{b}   =  \frac{\mathbf{a}^{T}\mathbf{b}}
                        {\mathbf{a}^{T}\mathbf{a}}
                        \mathbf{a}
                = \frac{5}{3}
                    \begin{bmatrix}
                        1   \\
                        1   \\
                        1
                    \end{bmatrix}
    \end{equation}
    Next we can verify that the residual $\mathbf{e}$ is perpendicular to
    $\mathbf{a}$  by verifying that $\mathbf{a}^{T}\mathbf{e}=0$, namely
    \begin{equation}
        \mathbf{e}  =
                    \mathbf{b} - P_{b} 
                    =
                    \begin{bmatrix}
                        -2/3    \\
                        1/3     \\
                        1/3
                    \end{bmatrix}
    \end{equation}
    thus, 
    \begin{equation}
        \mathbf{a}^{T}\mathbf{e} = 0
    \end{equation}

    We can use the same procedure as above the following problem. Starting
    with finding the projection of $\mathbf{b}$ onto $\mathbf{a}$,
    \begin{equation}
        P_{b} =
                \frac{\mathbf{a}^{T}\mathbf{b}}
                        {\mathbf{a}^{T}\mathbf{a}}
                        \mathbf{a}
                =
                -\frac{11}{11}
                \begin{bmatrix}
                    -1      \\
                    -3      \\
                    -1
                \end{bmatrix}
    \end{equation}
    Next we can verify that the residual is perpendicular to $\mathbf{a}$
    using the dot product, namely
    \begin{equation}
        \mathbf{e}  = \mathbf{b} - P_{b}
                        \begin{bmatrix}
                            0       \\
                            0       \\
                            0
                        \end{bmatrix}
    \end{equation}
    Thus,
    \begin{equation}
        \mathbf{a}^{T}\mathbf{e} = 0
    \end{equation}
\end{mdframed}
                    
\subsection{Part d.}
Project $\mathbf{b}$ onto the following subspaces
\begin{equation}
    \mathbf{b} =
    \begin{bmatrix}
        2       \\
        3       \\
        4  
    \end{bmatrix}
    \text{ onto span}
    \Bigg\{
        \begin{bmatrix}
            1       \\
            0       \\
            0
        \end{bmatrix}
        ,
        \begin{bmatrix}
            1       \\
            1       \\
            0
        \end{bmatrix}
        \Bigg\}
\end{equation}
and 
\begin{equation}
    \mathbf{b} =
    \begin{bmatrix}
        4       \\
        4       \\
        6  
    \end{bmatrix}
    \text{ onto span}
    \Bigg\{
        \begin{bmatrix}
            1       \\
            0       \\
            0
        \end{bmatrix}
        ,
        \begin{bmatrix}
            1       \\
            1       \\
            0
        \end{bmatrix},
        \begin{bmatrix}
            1       \\
            1       \\
            1
        \end{bmatrix}
        \Bigg\}
\end{equation}
\begin{mdframed}[style=MyFrame]
    The projection of $\mathbf{b}$ onto the span can be found using the
    $\mathbf{p} =
    \mathbf{A}(\mathbf{A}^{T}\mathbf{A})^{-1}\mathbf{A}^{T}\mathbf{b}$,
    namely 
    \begin{equation}
        \mathbf{p}  =
                    \mathbf{A}(\mathbf{A}^{T}\mathbf{A})^{-1}\mathbf{A}^{T}\mathbf{b}               
                    =
                    \begin{bmatrix}
                        2   \\
                        3   \\
                        0
                    \end{bmatrix}
    \end{equation}
    and
    \begin{equation}
        \mathbf{p}  =
                    \begin{bmatrix}
                        4   \\
                        4   \\
                        6
                    \end{bmatrix}
    \end{equation}
\end{mdframed}
\subsection{Part e.}
We seek to project $\mathbf{b}$ onto a subspace as given below. Explain why
the usual approach based on solving $\matA^{T} \matA \widehat{\mathbf{x}} =
\matA^{T} \mathbf{b}$ does not work, and propose a solution.
\begin{equation}
    \mathbf{b} =
    \begin{bmatrix}
        1   \\
        2
    \end{bmatrix}
    \text{ onto span}
    \Bigg\{
        \begin{bmatrix}
            3   \\
            4
        \end{bmatrix}
        ,
        \begin{bmatrix}
            9   \\
            12
        \end{bmatrix}
        \Bigg\}
\end{equation}
\begin{mdframed}[style=MyFrame]
    We start by recognizing that the two column vectors are linearly
    dependent therefore,
    \begin{equation}
        \text{span}
        \Bigg\{
            \begin{bmatrix}
                3   \\
                4
            \end{bmatrix}
            ,
            \begin{bmatrix}
                9   \\
                12
            \end{bmatrix}
            \Bigg\}
        =
        \Bigg\{
            \begin{bmatrix}
                3   \\
                4
            \end{bmatrix}
            \Bigg\}
    \end{equation}         
    \begin{equation}
        \mathbf{s}
        =
        \text{span}
        \Bigg\{
            \begin{bmatrix}
                3   \\
                4
            \end{bmatrix}
            \Bigg\}
    \end{equation}
    using
    \begin{equation}
        \mathbf{p}  = 
                    \frac{\mathbf{s}^{T}\mathbf{b}}
                            {\mathbf{s}^{T}\mathbf{s}}
                            \mathbf{s}
                    =
                    \frac{11}{25}
                    \begin{bmatrix}
                        3   \\
                        4
                    \end{bmatrix}
    \end{equation}
    Here solving 
    $\mathbf{A}^{T}\mathbf{A}\widehat{\mathbf{x}} =
    \mathbf{A}^{T}\mathbf{b}$, will not work since 
    $\mathbf{A}^{T}\mathbf{A}$ is not invertible because $\mathbf{A}$ has
    linearly dependent columns. The solution, is to find a basis of the
    subspace and use the basis vector to compute a least squares solution
    (i.e., projection) $\mathbf{B}^T\mathbf{B}
    \widehat{\mathbf{x}}=\mathbf{B}^{T}\mathbf{b}$, where $\mathbf{B}$ is
    the matrix with the basis vectors as columns.  Therefore we need to
    find the projection of $\mathbf{b}$ onto 


\end{mdframed}


        
    
    
