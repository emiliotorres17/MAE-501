\newcommand{\matA}{\mathbf{A}}
\newcommand{\matB}{\mathbf{B}}
\section{Exercise 2}
\subsection{Part a.}
Suppose $\mathbf{A}$ is $3 \times 4$ and $\mathbf{B}$ is $4 \times 5$ and
$\matA \matB =0$. So $N(\matA)$ contains $C(\matB)$. Prove from the
dimensions of $N(\matA)$ and $C(\matB)$ that $\text{rank}(\matA) +
\text{rank}(\matB) \leq 4$. 

\begin{mdframed}[style=MyFrame]
    since the $N(A)$ contains the $C(B)$ means the dimensions of $C(B) \leq
    \text{ dimension } N(A)$. Thus $\text{rank}(B) \leq 4 - \text{
        rank}(A)$.
\end{mdframed}

\subsection{Part b.}
For four non-zero vectors $\mathbf{r}$, $\mathbf{n}$, $\mathbf{c}$,
$\mathbf{l}$, in $\mathbb{R}^{2}$
\begin{enumerate}[label=(\alph*)]
    \item What are the conditions for those to be bases for the four
        fundamental subspaces for a $2 \times 2$ matrix?
    \item What is one possible matrix $\matA$?
\end{enumerate}

\subsection{Part c.}
Project the vector $\mathbf{b}$ onto the line through $\mathbf{a}$. Verify
that the residual $\mathbf{e}$ is perpendicular to $\mathbf{a}$.
\begin{equation}
    \mathbf{b} = 
    \begin{bmatrix}
        1       \\
        2       \\
        2
    \end{bmatrix}
    \hspace{2.0cm} 
    \mathbf{a} =
    \begin{bmatrix}
        1       \\
        1       \\
        1
    \end{bmatrix}
\end{equation}
and 
\begin{equation}
    \mathbf{b} = 
    \begin{bmatrix}
        1       \\
        3       \\
        1
    \end{bmatrix}
    \hspace{2.0cm} 
    \mathbf{a} =
    \begin{bmatrix}
        -1       \\
        -3       \\
        -1
    \end{bmatrix}
\end{equation}

\begin{mdframed}[style=MyFrame]
    We start by finding the projection of $\mathbf{b}$ onto $\mathbf{a}$,
    \begin{equation}
        P_{b}   =  \frac{\mathbf{a}^{T}\mathbf{b}}
                        {\mathbf{a}^{T}\mathbf{a}}
                        \mathbf{a}
                = \frac{5}{3}
                    \begin{bmatrix}
                        1   \\
                        1   \\
                        1
                    \end{bmatrix}
    \end{equation}
    Next we can verify that the residual $\mathbf{e}$ is perpendicular to
    $\mathbf{a}$  by verifying that $\mathbf{a}^{T}\mathbf{e}=0$, namely
    \begin{equation}
        \mathbf{e}  =
                    \mathbf{b} - P_{b} 
                    =
                    \begin{bmatrix}
                        -2/3    \\
                        1/3     \\
                        1/3
                    \end{bmatrix}
    \end{equation}
    thus, 
    \begin{equation}
        \mathbf{a}^{T}\mathbf{e} = 0
    \end{equation}

    We can use the same procedure as above the following problem. Starting
    with finding the projection of $\mathbf{b}$ onto $\mathbf{a}$,
    \begin{equation}
        P_{b} =
                \frac{\mathbf{a}^{T}\mathbf{b}}
                        {\mathbf{a}^{T}\mathbf{a}}
                        \mathbf{a}
                =
                -\frac{11}{11}
                \begin{bmatrix}
                    -1      \\
                    -3      \\
                    -1
                \end{bmatrix}
    \end{equation}
    Next we can verify that the residual is perpendicular to $\mathbf{a}$
    using the dot product, namely
    \begin{equation}
        \mathbf{e}  = \mathbf{b} - P_{b}
                        \begin{bmatrix}
                            0       \\
                            0       \\
                            0
                        \end{bmatrix}
    \end{equation}
    Thus,
    \begin{equation}
        \mathbf{a}^{T}\mathbf{e} = 0
    \end{equation}
\end{mdframed}
                    
\subsection{Part d.}
Project $\mathbf{b}$ onto the following subspaces
\begin{equation}
    \mathbf{b} =
    \begin{bmatrix}
        2       \\
        3       \\
        4  
    \end{bmatrix}
    \text{ onto span}
    \Bigg\{
        \begin{bmatrix}
            1       \\
            0       \\
            0
        \end{bmatrix}
        ,
        \begin{bmatrix}
            1       \\
            1       \\
            0
        \end{bmatrix}
        \Bigg\}
\end{equation}
and 
\begin{equation}
    \mathbf{b} =
    \begin{bmatrix}
        4       \\
        4       \\
        6  
    \end{bmatrix}
    \text{ onto span}
    \Bigg\{
        \begin{bmatrix}
            1       \\
            0       \\
            0
        \end{bmatrix}
        ,
        \begin{bmatrix}
            1       \\
            1       \\
            0
        \end{bmatrix},
        \begin{bmatrix}
            1       \\
            1       \\
            1
        \end{bmatrix}
        \Bigg\}
\end{equation}
\begin{mdframed}[style=MyFrame]
    We can start by defining the projection onto the span as,
    \begin{equation}
        P_{b} =  
            \frac{\mathbf{v}_{1}^{T} \mathbf{b}}
            {\mathbf{v}_{1} \mathbf{v}_{1}}
            \mathbf{v}_{1}
            +
            \frac{\mathbf{v}_{2}^{T} \mathbf{b}}
            {\mathbf{v}_{2} \mathbf{v}_{2}}
            \mathbf{v}_{2}
    \end{equation}
    where $\mathbf{v}_{1}$ and $\mathbf{v}_{2}$ are the column vectors of
    the span. Substituting in the column vectors of the span gives,
    \begin{equation}
        P_{b} = 
            \frac{2}{1}
            \begin{bmatrix}
                1       \\
                0       \\
                0
            \end{bmatrix}
            +
            \frac{5}{2}
            \begin{bmatrix}
                1       \\
                1       \\
                0
            \end{bmatrix}
    \end{equation}
    Thus,
    \begin{equation}
        P_{b} =
        \begin{bmatrix}
            9/2     \\
            5/2     \\
            0
        \end{bmatrix}
    \end{equation}
    Next we can repeat the process for part two of the question,
    \begin{equation}
        P_{b} = 
            \frac{\mathbf{v}_{1}^{T} \mathbf{b}}
            {\mathbf{v}_{1} \mathbf{v}_{1}}
            \mathbf{v}_{1}
            +
            \frac{\mathbf{v}_{2}^{T} \mathbf{b}}
            {\mathbf{v}_{2} \mathbf{v}_{2}}
            \mathbf{v}_{2}
            +
            \frac{\mathbf{v}_{3}^{T} \mathbf{b}}
            {\mathbf{v}_{3} \mathbf{v}_{3}}
            \mathbf{v}_{3}
    \end{equation}
    which gives,
    \begin{equation}
        P_{b} = 
            \frac{4}{1}
            \begin{bmatrix}
                1       \\
                0       \\
                0
            \end{bmatrix}
            +
            \frac{8}{2}
            \begin{bmatrix}
                1       \\
                1       \\
                0
            \end{bmatrix}
            +
            \frac{14}{3}
            \begin{bmatrix}
                1       \\
                1       \\
                1
            \end{bmatrix}
    \end{equation}
    Thus,
    \begin{equation}
        P_{b} =
        \begin{bmatrix}
            38/3    \\
            26/3    \\
            14/3
        \end{bmatrix}
    \end{equation}
\end{mdframed}
\subsection{Part e.}
We seek to project $\mathbf{b}$ onto a subspace as given below. Explain why
the usual approach based on solving $\matA^{T} \matA \widehat{\mathbf{x}} =
\matA^{T} \mathbf{b}$ does not work, and propose a solution.
\begin{equation}
    \mathbf{b} =
    \begin{bmatrix}
        1   \\
        2
    \end{bmatrix}
    \text{ onto span}
    \Bigg\{
        \begin{bmatrix}
            3   \\
            4
        \end{bmatrix}
        ,
        \begin{bmatrix}
            9   \\
            12
        \end{bmatrix}
        \Bigg\}
\end{equation}
