\section{Exercise 3}
%-------------------------------------------------------------------------%
% Part a.                                                                 %
%-------------------------------------------------------------------------%
\subsection{Part a.}
Find the matrix that transforms a vector $(x_{1}, x_{2}, x_{3})$ into 
$(x_{2}, x_{3}, x_{1})$ in 3D. Verify that this is a rotation matrix and
give its axis and angle of rotation.
%-------------------------------------------------------------------------%
% Part b.                                                                 %
%-------------------------------------------------------------------------%
\subsection{Part b.}
Find the orthonormal vector $\mathbf{q}_{1}$, $\mathbf{q}_{2}$, and 
$\mathbf{q}_{3}$ such that $\mathbf{q}_{1}$ and $\mathbf{q}_{2}$ span the
column space of $\mathbf{A}$.
\begin{equation}
    \mathbf{A} =
    \begin{bmatrix}
        1       &       1       \\
        2       &       -1      \\
        -2      &       4       
    \end{bmatrix}
\end{equation}
\begin{enumerate}[label=(\arabic*)]
    \item Which of the four fundamental subspaces contain $\mathbf{q}_{1}$?
    \item Solve $\mathbf{A}\mathbf{x} = (4,2,2)$ by least squares?
\end{enumerate}
%-------------------------------------------------------------------------%
% Part c.                                                                 %
%-------------------------------------------------------------------------%
\subsection{Part c.}
Find an orthonormal basis for the column space of $\mathbf{A}$ and compute
the projection of $\mathbf{b}$ onto that column space:
\begin{equation}
    \mathbf{A} =
    \begin{bmatrix}
        1       &       -2  \\
        1       &       0   \\
        1       &       1   \\
        4       &       2   \\
    \end{bmatrix}
\end{equation}
\begin{mdframed}
    We start by finding $C(A)$ using $\text{rref}(A)$,
    \begin{equation}
        \text{rref}(A) = 
        \begin{bmatrix}
            1       &       0   \\
            0       &       1   \\
            0       &       0   \\
            0       &       0   \\
        \end{bmatrix}
    \end{equation}
    Therefore,
    \begin{equation}
        C(A) =
        \begin{bmatrix}
        1       &       -2  \\
        1       &       0   \\
        1       &       1   \\
        4       &       2   \\
        \end{bmatrix}
        = 
        \begin{bmatrix}
            c_{1}   &   c_{2}
        \end{bmatrix}
    \end{equation}
    Next, we can perform the Gram-Schmidt process to find an orthogonal
    basis, namely
    \begin{equation}
        v_{1} = c_{1} =   
        \begin{bmatrix}
            1       \\
            1       \\
            1       \\
            4
        \end{bmatrix}
    \end{equation}
    \begin{equation}
        v_{2}   = c_{2} - \frac{v_{1}^{T}c_{2}}{v_{1}^{T}v_{1}} v_{1}
                = 
                \begin{bmatrix}
                    -3      \\
                    -1      \\
                    0       \\
                    2
                \end{bmatrix}
    \end{equation}
    Lastly, we can divide through by the magnitudes to normalize the
    vectors
    \begin{equation}
        \mathbf{q}_{1}  = \frac{v_{1}}{\norm{v_{1}}
                        = \frac{}

%-------------------------------------------------------------------------%
% Part d.                                                                 %
%-------------------------------------------------------------------------%
\subsection{Part d.}
Find orthogonal vectors $\mathbf{A}$, $\mathbf{B}$, and $\mathbf{C}$ by
Gram-Schmidt from
\begin{equation}
    a = 
    \begin{bmatrix}
        1       \\
        1       \\
        2
    \end{bmatrix}
    \text{ and }
    b = 
    \begin{bmatrix}
        1       \\
        -1      \\
        0
    \end{bmatrix}
    \text{ and }
    c = 
    \begin{bmatrix}
        1       \\
        0       \\
        4
    \end{bmatrix}
\end{equation}
\begin{mdframed}[style=MyFrame]
    Begin by choosing $\mathbf{A}=\mathbf{a}$. Next, start with
    $\mathbf{b}$ and subtract its projection along $\mathbf{A}$ using 
    \begin{equation}
        \mathbf{B} =  \mathbf{b} 
                        - \frac{\mathbf{A}^{T} \mathbf{b} }
                                {\mathbf{A}^{T} \mathbf{A}}
                                \mathbf{A}
    \end{equation}
    which gives
    \begin{equation}
        \mathbf{B} = 
        \begin{bmatrix}
            1       \\
            -1      \\
            0
        \end{bmatrix}
        -
        \frac{0}{6}
        \begin{bmatrix}
            1       \\
            1       \\
            2
        \end{bmatrix}
        =
        \begin{bmatrix}
            1       \\
            -1      \\
            0
        \end{bmatrix}
    \end{equation}
    Continuing we can perform the next Gram-Schmidt step by subtracting off
    the projections of $\mathbf{c}$ from $\mathbf{A}$ and $\mathbf{B}$,
    using
    \begin{equation}
        \mathbf{C} =  \mathbf{c} 
                        - \frac{\mathbf{A}^{T} \mathbf{c} }
                                {\mathbf{A}^{T} \mathbf{A}}
                                \mathbf{A}
                        - \frac{\mathbf{B}^{T} \mathbf{c} }
                                {\mathbf{B}^{T} \mathbf{B}}
                                \mathbf{B}
    \end{equation}
    giving
    \begin{equation}
        \mathbf{C} = 
        \begin{bmatrix}
            1       \\
            0       \\
            4
        \end{bmatrix}
        -
        \frac{9}{6}
        \begin{bmatrix}
            1       \\
            1       \\
            2
        \end{bmatrix}
        -
        \frac{1}{2}
        \begin{bmatrix}
            1       \\
            -1      \\
            0
        \end{bmatrix}
        =
        \begin{bmatrix}
            -1      \\
            -1      \\
            1
        \end{bmatrix}
    \end{equation}
    Lastly we can normalize the vectors by dividing by their magnitudes,
    \begin{equation}
        \frac{\mathbf{A}}{\norm{\mathbf{A}}} =
            \frac{1}{\sqrt{6}}
            \begin{bmatrix}
                1       \\
                1       \\
                2
            \end{bmatrix}
    \end{equation}
    \begin{equation}
        \frac{\mathbf{B}}{\norm{\mathbf{B}}} =
            \frac{1}{\sqrt{2}}
            \begin{bmatrix}
                1       \\
                -1      \\
                0
            \end{bmatrix}
    \end{equation}
    \begin{equation}
        \frac{\mathbf{C}}{\norm{\mathbf{C}}} =
            \frac{1}{\sqrt{3}}
            \begin{bmatrix}
                -1      \\
                -1      \\
                1
            \end{bmatrix}
    \end{equation}
\end{mdframed}
%-------------------------------------------------------------------------%
% Part e.                                                                 %
%-------------------------------------------------------------------------%
\subsection{Part e.}
Find $\mathbf{q}_{1}$, $\mathbf{q}_{2}$, and $\mathbf{q}_{3}$ (orthonormal)
as combination of $\mathbf{a}$, $\mathbf{b}$, and $\mathbf{c}$ (independent
columns). Then write $\mathbf{A}$ as $\mathbf{QR}$.
\begin{equation}
    \mathbf{A} =
    \begin{bmatrix}
        1       &       2       &   4   \\
        0       &       0       &   5   \\
        0       &       3       &   6
    \end{bmatrix}
\end{equation}

\begin{mdframed}[style=MyFrame]
    We start by finding $\mathbf{q}_{1}$, $\mathbf{q}_{2}$, and
    $\mathbf{q}_{3}$ using
    \begin{equation}
        \mathbf{q}_{1} = \frac{A}{\norm{A}}
    \end{equation}
    \begin{equation}
        \mathbf{q}_{2} = \frac{B}{\norm{B}}
    \end{equation}
    \begin{equation}
        \mathbf{q}_{3} = \frac{C}{\norm{C}}
    \end{equation}
    where $\mathbf{A}$, $\mathbf{B}$, and $\mathbf{C}$ are found from the
    Gram-Schmidt process, namely
    \begin{equation}
        \mathbf{A} = \mathbf{a} = 
        \begin{bmatrix}
            1   \\
            0   \\
            0   \\
        \end{bmatrix}
    \end{equation}
    \begin{equation}
        \mathbf{B} = 
        \mathbf{\mathbf{b}} 
        - \frac{ \mathbf{A}^{T} \mathbf{b} }{ \mathbf{A}^{T}\mathbf{A} }\mathbf{A}
        =
        \begin{bmatrix}
            0   \\
            0   \\
            3   \\
        \end{bmatrix}
    \end{equation}
    \begin{equation}
        \mathbf{C} = 
        \mathbf{c} 
        - \frac{ \mathbf{A}^{T} \mathbf{c} }{ \mathbf{A}^{T}\mathbf{A} }\mathbf{A}
        - \frac{ \mathbf{B}^{T} \mathbf{c} }{ \mathbf{B}^{T}\mathbf{B} }\mathbf{B}
        =
        \begin{bmatrix}
            0   \\
            5   \\
            0   \\
        \end{bmatrix}
    \end{equation}
    Therefore,
    \begin{equation}
        \mathbf{q}_{1} =  
        \begin{bmatrix}
            1   \\
            0   \\
            0
        \end{bmatrix}
    \end{equation}
    \begin{equation}
        \mathbf{q}_{2} =  
        \begin{bmatrix}
            0   \\
            0   \\
            1
        \end{bmatrix}
    \end{equation}
    \begin{equation}
        \mathbf{q}_{3} =  
        \begin{bmatrix}
            0   \\
            1   \\
            0
        \end{bmatrix}
    \end{equation}
    Lastly we can find $\mathbf{R}$ using,
    \begin{equation}
        \mathbf{R} =
        \begin{bmatrix}
            \mathbf{q}_{1}^{T}\mathbf{a}    &   \mathbf{q}_{1}^{T}\mathbf{b}    &   \mathbf{q}_{1}^{T}  \mathbf{c}  \\
            0                               &   \mathbf{q}_{2}^{T}\mathbf{b}    &   \mathbf{q}_{2}^{T}  \mathbf{c}  \\
            0                               &   0                               &   \mathbf{q}_{3}^{T}  \mathbf{c}  \\
        \end{bmatrix}
    \end{equation}
    Thus,
    \begin{equation}
        \begin{bmatrix}
            \mathbf{a}      &       \mathbf{b}      &   \mathbf{c}  
        \end{bmatrix}
        = 
        \begin{bmatrix}
            \mathbf{q}_{1}  &       \mathbf{q}_{2}  &   \mathbf{q}_{3}  
        \end{bmatrix}
        \begin{bmatrix}
            \mathbf{q}_{1}^{T}\mathbf{a}    &   \mathbf{q}_{1}^{T}\mathbf{b}    &   \mathbf{q}_{1}^{T}  \mathbf{c}  \\
            0                               &   \mathbf{q}_{2}^{T}\mathbf{b}    &   \mathbf{q}_{2}^{T}  \mathbf{c}  \\
            0                               &   0                               &   \mathbf{q}_{3}^{T}  \mathbf{c}  \\
        \end{bmatrix}
        =
        \underbrace{
            \begin{bmatrix}
                1   &   0   &   0   \\
                0   &   0   &   1   \\
                0   &   1   &   0   \\
            \end{bmatrix}
        }_{=Q}
        \underbrace{
            \begin{bmatrix}
                1   &   2   &   4   \\
                0   &   3   &   6   \\
                0   &   0   &   5   \\
            \end{bmatrix}
        }_{=R}
    \end{equation}
\end{mdframed}
